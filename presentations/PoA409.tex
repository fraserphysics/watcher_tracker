\documentclass{beamer}
\usepackage{amsmath,amsfonts}
\usepackage[pdftex]{rotating}
\newcommand{\argmax}{\operatorname*{argmax}}
\newcommand{\ti}[2]{{#1}{(#2)}}                         % Index
\newcommand{\ts}[3]{{#1}_{#2}^{#3}} % Time series
\title{Multi-Target Tracking from Dynamic Programming}
\author{Andy Fraser}
\date{March 27, 2008}

%\usetheme{Pittsburgh}
\usetheme{default}
\usefonttheme[]{serif}
\begin{document}
\frame{\titlepage}

%\frame{\tableofcontents}

\frame{
  \frametitle{Big Picture}
  \begin{description}
  \item[Goal:] Low Frame-rate Video $\Rightarrow$ Vehicle Tracks
  \item[Method:] \hspace{1em}
    \begin{itemize}
    \item Apply detector to frame  $\Rightarrow$ \emph{hits}$(t)$
    \item Connect \emph{hits} $\Rightarrow$ \emph{tracks}
    \end{itemize}
  \end{description}
}

\frame{
  \frametitle{For now}
  \begin{description}
  \item[Input:] List of \textbf{hits} $\left\{ y_i(t) \right\}$ for
    frame at each time $t$
    \begin{equation*}
      y_i(t) = \text{Position \& Attributes}
    \end{equation*}
  \item[Output:] List of \textbf{trajectories} for each
    \textbf{target}
    \begin{equation*}
      \left\{ x_j(t): T_i
        \leq t \leq T_f \right\}
    \end{equation*}
    where
    \begin{equation*}
      x_j(t) = \text{Position \& Velocity}
    \end{equation*}
  \end{description}
(Future: Track info.\ $\rightarrow$ detector.  Use appearance,
orientation, illumination)
}

\frame{
  \frametitle{Solution Strategy}
  \begin{description}
  \item[Want to characterize:]
    \begin{equation*}
      P(\ts{x}{1}{T}|\ts{y}{1}{T}) = \sum_{\ts{A}{1}{T}}  P(\ts{x}{1}{T},
      \ts{A}{1}{T}|\ts{y}{1}{T})
    \end{equation*}
    But density has $N^T$ components
  \item[Approximate:]  with sequences of states and associations:
    \begin{equation}
      \label{eq:bear}
      \ts{{\hat x}}{1}{T},\ts{{\hat A}}{1}{T} =
      \argmax_{\ts{x}{1}{T},\ts{A}{1}{T}}   P(\ts{x}{1}{T},
      \ts{A}{1}{T}|\ts{y}{1}{T})
    \end{equation}
    Like Kalman filtering and Viterbi decoding
  \item[State space model:] Linear target dynamics, Gaussian accelerations:
    $x_j(t+1) = A \cdot x_j(t) + \eta(t)$
  \item[Similar to:] Multi-hypothesis tracking.  (MHT: DB Reid IEEE
    Trans AC 1979)
  \end{description}
}

\frame{
  \frametitle{Novelty?}
  \begin{itemize}
  \item Variant of Viterbi algorithm, ie, dynamic programming
  \item Use entire sequence $\bf{y}_1^T$ to guess $\bf{\hat x}_1^T$
  \item At time $t$, \emph{state} $\bf{s}(t)$ has two components:
    \begin{description}
    \item[continuous $\bf{x}(t)$:] (position, velocity) of each target
    \item[discrete $\bf{a}(t)$:] association of targets to hits
    \end{description}
  \end{itemize}
}

\frame{
  \frametitle{Notation for Viterbi decoding}
\begin{align*}
  u(\ts{s}{1}{t}) & \quad \text{Utility of state sequence }
  \ts{s}{1}{t}\\
  & \quad \equiv \log \left( P(\ts{y}{1}{t},\ts{s}{1}{t}) \right)
  \\
  \nu(s,t) & \quad \text{Utility of best sequence ending with }
  \ti{s}{t} = s \\
  &  \quad \equiv \max_{\ts{s}{1}{t}:\ti{s}{t}=s} u(\ts{s}{1}{t}) \\
  B(s',t) & \quad \text{Best predecessor state given } \ti{s}{t+1}=s'\\
  & \quad \equiv \argmax_{s} \max_{\ts{s}{1}{t+1}:\ti{s}{t}=s
    \&\ti{s}{t+1}=s'} u(\ts{s}{1}{t+1})
\end{align*}
Insights:
\begin{itemize}
\item Calculate $\nu(s,t)$ and $B(s,t)$ $\forall s \& t$ recursively
\item For most $\ts{s}{1}{t}$, never calculate $u(\ts{s}{1}{t})$
\item Complexity linear in number of frames $T$
\end{itemize}
}

\frame{
  \frametitle{Viterbi Algorithm Picture}
  \begin{columns}
    \begin{column}[l]{0.5\textwidth}
      At $t=0$:
      \begin{equation*}
        \vspace{2cm}
        \nu(s,0) = \log\left( P(s) P(y(0)|s) \right)
      \end{equation*}
    \end{column}
    \begin{column}[r]{0.3\textwidth}
      \resizebox{0.99\textwidth}{!}{\includegraphics{viterbi.pdf}}
    \end{column}
  \end{columns}
      At $t=1$:
      \begin{align*}
        \nu(s,1) &= {\color{red}{\max_{s'}}} \left( \nu(s',0) +
          \log\left( P(s|s') 
            P(y(1)|s) \right) \right) \\
        B(s,1) &=  {\color{red}{\argmax_{s'}}} \left( \nu(s',0) +
          \log\left( P(s|s') 
            P(y(1)|s) \right) \right)
      \end{align*}
      Need {\color{magenta}{only keep}} {\color{cyan}{$\nu(s,t)$}} and
      {\color{cyan}{$B(s,t)$}} $\forall~s,t$

}

\frame{ \frametitle{Pruning and Approximation}
  Considering all associations: Complexity $= N!T$\\
  Improve scaling by:
  \begin{itemize}
  \item Cluster hits via targets
  \item Cutoff associations for each cluster by requiring
    \begin{itemize}
    \item $N_{\bf a} < N_{\text{max}}$
    \item $\nu({\bf a}) > \nu_{\text{best}} - \Delta_{\text{max}}$
    \end{itemize}
  \item Prune sequences that match for 5 frames
  \end{itemize}

}

\frame{
  \frametitle{Code Demo}
  \newcommand{\entrywidth}{4em}
  \begin{center}
    \begin{tabular}[t]
      {|p{\entrywidth}|p{\entrywidth}|p{\entrywidth}|p{\entrywidth}|}
      \hline
      Models & Variable Visibility & False Alarms & Variable Num.\ Targets \\
      \hline 1   &   &   & \\
      \hline 2   & Y &   & \\
      \hline 3   & Y & Y & \\
      \hline 4   & Y & Y & Y \\
      \hline ABQ &   &   & Y \\
      \hline
    \end{tabular}
  \end{center}
  \href{run:start_View.sh}{\color{red}{Start GUI}}
}

\frame{
  \frametitle{Challenging Scene}
  \href{run:movie_script1.sh}{
    \framebox[\textwidth]{\rule{0cm}{6cm}orig.mpg here}
  }
}

\frame{ \frametitle{Grade Against Marked Ground Truth}
  \href{run:movie_script2.sh}{
    \framebox[\textwidth]{\rule{0cm}{6cm}markup.mpg here}
  }
}

\frame{
  \frametitle{MHT}
  \href{run:movie_script3.sh}{
    \framebox[\textwidth]{\rule{0cm}{6cm}track1.mpg here}
  }
}

% \frame{
%   \frametitle{Pruning Techniques}
%   \begin{itemize}
%   \item Cluster hits, and set threshold distance
%   \item 10 best associations for each cluster
%   \item Prune sequences that match for 5 frames
%   \end{itemize}
%   \newcommand{\fudgeA}{0.4}
%   \parbox{\fudgeA\textwidth}{\center{Hits vs Time}\\
%     \resizebox{\fudgeA\textwidth}{!}{\includegraphics{figA3.png}}}
%   \hspace{-0.02\textwidth}
%   \parbox{\fudgeA\textwidth}{\center{Trajectories}\\
%     \resizebox{\fudgeA\textwidth}{!}{\includegraphics{figB3.png}}}
% }

\frame{
  \frametitle{Status:}
  \begin{itemize}
  \item Good results for sparse traffic from extreme pruning
  \item For moderate traffic, MHT and multiple modes helpful
  \item Can't do dense traffic
  \item Current code implements Hungarian/Murty algorithm and
    clustering.
  \end{itemize}
}

\frame{
  \frametitle{Next steps:}
  \begin{itemize}
  \item Make model parameters depend on position
  \item Exploit attributes, eg, size/shape
  \item Make motion model depend on orientation
  \item Flag difficult decisions
  \item Spend computational budget on \emph{important} tracks
  \end{itemize}
}

\end{document}

% To get evince to run shell scripts put this line in
% ~/.local/share/applications/mimeinfo.cache:

% application/x-shellscript=bash-usercreated.desktop

% In bash-usercreated.desktop put:

%[Desktop Entry]
%Encoding=UTF-8
%Name=bash
%MimeType=application/x-shellscript;
%Exec=/bin/bash %f
%Type=Application
%Terminal=false
%NoDisplay=true


%%%---------------
%%% Local Variables:
%%% eval: (TeX-PDF-mode)
%%% End:

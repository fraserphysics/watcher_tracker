\documentclass{article}

% These are the contributed presentation proposals that I submitted.
% The first one is a reduced version of the minisymposium proposal
% that I did not submit.  The second one is for a poster.

\begin{document}
\begin{description}
\item[MS28:] Monday 8:30 AM Wasatch A.  Mine 9-9:25
\item[Title:] Video target tracking: Uncertain associations, dynamical
  models, and state estimation
\item[Abstract:] % 75 word limit
  Multi-target tracking algorithms for exploiting low frame rate video
  of urban traffic must contend with a combinatorial number of
  possible associations between objects detected in successive frames.
  While the mapping of sequences of non-invertible stochastic
  measurements to state sequences is familiar to the dynamical systems
  community, the combinatorial association aspect of this application
  may not be.  I use analogies to ideas the dynamical systems
  community have long considered to explain current algorithms.
\item[Author Thing:] Monday 7:45 PM to 8:30
% Next is the poster proposal that I submitted for the meeting
\item[Poster session:]  8:30 PM Tuesday
\item[Title:] Fidelity criteria and entropy
\item[Abstract:] % 75 word limit
  I propose expected log likelihood per time (the cross entropy rate)
  as a fidelity measure for dynamical models.  By this measure the
  performance of any model fit to a \emph{true} system is bound to be
  worse than Kolmogorov and Sinai's entropy rate.  I illustrate with a
  sequence of more and more complex hidden Markov models that
  approximate the Lorenz system and find that approaching the KS bound
  requires millions of discrete states.
\end{description}

\end{document}

%%% Local Variables:
%%% eval: (TeX-PDF-mode)
%%% End:

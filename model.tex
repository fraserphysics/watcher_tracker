\documentclass[12pt]{article}

\usepackage{amsmath,amsfonts}
\usepackage{graphicx}
\usepackage{showlabels}

\newcommand{\normal}[2]{{\cal N}(#1,#2)}
\newcommand{\NormalE}[3]{{\mathcal{N}}\left.\left(#1,#2\right)\right|_{#3}}
\newcommand{\xdot}{{\dot x}}
\renewcommand{\th}{^{\text{th}}}
\newcommand{\field}[1]{\mathbb{#1}}
\newcommand{\REAL}{\field{R}}
\newcommand{\RATIONAL}{\field{Q}}
\newcommand{\INTEGER}{\field{Z}}
\newcommand{\EV}[2]{\field{E}_{#1}\left[#2\right]}
\newcommand{\M}{{\cal M}}
\newcommand{\transpose}{^\top}
\newcommand{\os}[4]{{\left[ #1(#2) \right]}_{#3}^{#4}} % Object sequence
\newcommand{\ti}[2]{{#1}{(#2)}}                         % Index
\newcommand{\ts}[4]{\os{#1}{#2}{#2=#3}{#4}} % Time series
%\newcommand{\ts}[4]{{\left[ #1(#2) \right]}_{#2=#3}^{#4}} % Sequence
\newcommand{\argmin}{\operatorname*{argmin}}
\newcommand{\argmax}{\operatorname*{argmax}}
\newcommand{\cS}{{\cal S}}

\title{Model Based Motion Detection}
\author{Andy Fraser}

\begin{document}
\maketitle

\section{Introduction}
\label{sec:introduction}

Frame by frame detection/tagging could use the result of a \emph{filtered}
forecast, ie,
\begin{equation*}
  P(S(t)|y_1^{t-1}).
\end{equation*}
Note that the best or \emph{decoded} track ie,
\begin{align*}
  \hat s_1^T &\equiv \argmax_{s_1^T} P(s_1^T|y_1^{t-1}) \\
            &=  \argmax_{s_1^T} P(s_1^T,y_1^{t-1})
\end{align*}
is different than the sequence of filtered forecasts.

\subsection{Filtering}
\label{sec:filtering}

With the definitions
\begin{align}
  \label{def:alpha}
  \alpha_t &\equiv P(\ti{s}{t}|\ts{y}{\tau}{1}{t}) \text{ The updated distribution} \\
  f_t &\equiv P(\ti{s}{t}|\ts{y}{\tau}{1}{t-1}) \text{ The forecast
    distribution} \\
  \gamma_t &= P(\ts{y}{\tau}{1}{t}) \text{ The incremental likelihood}
\end{align}
and applying a model characterized by the distributions
\begin{align}
  \label{def:alpha0}
  &\alpha_0  &&\text{The state prior} \\
  &P(\ti{s}{t+t}|\ti{s}{t}) &&\text{The state transition probability} \\
  &P(\ti{y}{t}|\ti{s}{t}) &&\text{The conditional observation probability}
\end{align}
I can write recursive filtering as follows:
\begin{align}
  f_t &= \EV{\alpha_{t-1}} {P(\ti{s}{t}|\ti{s}{t-1})} \\
  \gamma_t &= \EV{\ti{f}{t}} {P(\ti{y}{t}|\ti{s}{t})} \\
  \alpha_t &= \frac{f_t P(\ti{y}{t}|\ti{s}{t})}{\gamma_t} \\
\end{align}

\subsection{Decoding}
\label{sec:decoding}

I will use the following definitions to describe decoding:
\begin{align*}
  u(\ts{s}{\tau}{1}{t}) &\equiv \log \left(
    P(\ts{y}{\tau}{1}{t},\ts{s}{\tau}{1}{t} \right) \\ & \quad
  \text{The utility of the state sequence } \ts{s}{\tau}{1}{t}\\
  \nu_t(s) & \equiv \max_{\ts{s}{\tau}{1}{t}:\ti{s}{t}=s}
  u(\ts{s}{\tau}{1}{t}) \\
  & \quad \text{The utility of the best sequence ending in } s \\
  B_t(s) &\equiv \argmax_{s'}
  \argmax_{\ts{s}{\tau}{1}{t}:\ti{s}{t-1}=s' \&\ti{s}{t}=s}
  u(\ts{s}{\tau}{1}{t}) \\
  & \quad \text{The best predecessor state for } s
\end{align*}
I can evaluate the functions $B_t$ and $\nu_t$ recursively as follows
\begin{align*}
  B_t(s) &= \argmax_{s'} \left( \nu_{t-1}(s') + \log(P(s|s') ) \right) \\
  \nu_t(s) &= \nu_{t-1}(B_t(s)) + \log(P(s|B_t(s)) )  + \log(P(\ti{y}{t}|s) ).
\end{align*}
Given the functions $B_t$ and $\nu_t$ $\forall t\in[1,\ldots,T]$, the
following procedure decodes the best sequence of states $ \ts{\hat
  s}{\tau}{1}{T}$ from a sequence of observations $
\ts{y}{\tau}{1}{T}$:
\begin{align*}
  {\ti{{\hat s}}{T}} &= \argmax_s \nu_T(s) \\
  & \text{for } t \text{ from } T-1 \text{ to } 1: \\
  & \quad \ti{\hat s}{t} = B_t( \ti{\hat s}{t+1})
\end{align*}

\section{Model Version One}
\label{sec:model1}

In this section I describe the basic model \emph{MVI}.  I'll develop
algorithms and write code for \emph{MVI} for the baseline prototype of
the watcher project.  Later, I will compare proposed enhancements
against the performance of \emph{MVI}.

The number of moving objects is a constant $N$ over time.  At each
time $t$, the separate objects $\ti{s}{t,j}$ that constitute the state
evolve independently of each other.  The following equations describe
$P(\ti{s}{t+1,j}|\ti{s}{t,j})$, ie, the dynamics of each object:
\begin{subequations}
  \label{eq:dynamics}
  \begin{align}
    \ti{s}{t+1,j} &= A  \ti{s}{t,j} + \ti{\epsilon}{t,j}, &
    \ti{\epsilon}{t,j} &\sim \normal{0}{\Sigma_{D}} \text{ iid} \\
    A &= \begin{bmatrix}
      1 & 0 & 1 & 0 \\
      0 & 1 & 0 & 1 \\
      0 & 0 & 1 & 0 \\
      0 & 0 & 0 & 1
    \end{bmatrix} &
    \Sigma_{D} &= \begin{bmatrix}
      \sigma^2_{xD} & 0 & 0 & 0 \\
      0 & \sigma^2_{xD} & 0 & 0 \\
      0 & 0 & \sigma^2_{\xdot D} & 0 \\
      0 & 0 & 0 & \sigma^2_{\xdot D}
    \end{bmatrix}
  \end{align}
\end{subequations}
Each object has the following components at time $t$:
\begin{align*}
  &&s_{0}(t,j) & \text{ horizontal position} \\
  &&s_{1}(t,j) & \text{ vertical position} \\
  &&s_{2}(t,j) & \text{ horizontal velocity} \\
  &&s_{3}(t,j) & \text{ vertical velocity}
\end{align*}
In addition to the moving objects, a complete description of a state
$\ti{s}{t}$ includes a map $\ti{M}{t}\in \M$, where $\M$ is the set of
permutations of $N$ items.  I assume that $\ti{M}{t}$ is distributed
uniformly and independently of everything else, ie,
\begin{equation*}
  P(\ti{M}{t}) = \frac{1}{\left| \M\right|} =  \frac{1}{N!}.
\end{equation*}

An observation vector consists of locations
$\ti{y}{t}=\os{y}{t,i}{i=1}{N}$.  The probability that state
$\ti{s}{t}$ would produce observation $\ti{y}{t}$ is\footnote{This
  observation model describes indistinguishable observations.  An
  easier alternative would be objects that had different colors, or
  even easier, objects that had observable unique tags.}
\begin{equation}
  \label{eq:ob}
  P(\ti{y}{t}|\ti{s}{t}) =
    \frac{1}{\left|\M \right|} \sum_{M \in \M}
    \prod_{i=1}^{N} P(\ti{y}{t,M(i)}|\ti{s}{t,i}).
\end{equation}
where
\begin{equation*}
  \ti{y}{t,i}|\ti{s}{t,j} \sim
  \begin{cases}
    \NormalE{\ti{s}{t,j}}{\Sigma_o}{\ti{y}{t,i}} & i = M(j) \\
    \text{Uniform} & \text{Otherwise}
  \end{cases}
\end{equation*}
and
\begin{equation*}
  \Sigma_o = \begin{bmatrix} \sigma_{xo}^2 & 0 \\ 0 &
    \sigma_{xo}^2 \end{bmatrix}.
\end{equation*}
In summary, aside from the initial state distribution, the model has
the following 3 degrees of freedom:
\begin{center}
  \begin{tabular}{|cp{15em}c|}
    \hline
    Symbol & Description & Degrees of freedom \\
    \hline
    $\Sigma_D$ & Dynamical noise & 2 \\
    $\Sigma_o$ & Observation noise & 1 \\
    \hline
  \end{tabular} 
\end{center}

\subsection{Approximation Schemes for Filtering}
\label{sec:approximation.filter}

With luck (I have not been so lucky with this application) one might
find a parametric form for the state prior $\alpha(0)$ (see
Eqn.~\eqref{def:alpha0}) that, combined with $\ti{y}{1}$, yields an
expression for $\alpha(1)$ that has the same parametric form.  Such a
form is called a \emph{conjugate family}.  Two particularly simple
cases are discrete hidden Markov models and Kalman filters.

Here, I will analyze the use of a Gaussian for $\alpha(0)$.  From that
choice it follows that each subsequent $\ti{\alpha}{t}$ is much more
complex.  Thus any actual implementation that starts with such an
$\alpha(0)$ must use simplifying approximations.  I will conclude the
section by considering a few such approximations.

Let me suppose that the initial state distribution is composed of
independent Gaussians, ie,
\begin{align*}
  \ti{\alpha}{0} &= P_{\os{S}{0,j}{j=1}{N} }\\
  &= \prod_{j=1}^{N} P_{\ti{S}{0,j}} \\
  &= \prod_{j=1}^{N} \normal{\mu_j}{\Sigma_j}.
\end{align*}
To find the forecast $\ti{f}{1}$, I can apply the state dynamics
\eqref{eq:dynamics} to each object independently with the result
\begin{align}
  \tilde \mu_j &= A \mu_j \\
  \tilde \Sigma_j &= A \Sigma_j A\transpose + \Sigma_D \\
  \label{eq:defF}
  (\tilde \mu_j, \tilde \Sigma_j ) &\equiv F(\mu_j,\Sigma_j) \\
  \label{eq:f1}
  \ti{f}{1} &= \prod_{j=1}^{N} {\cal N}(F(\mu_j,\Sigma_j)),
\end{align}
where Eqn.~\eqref{eq:defF} defines the map $F$ from a distribution at
time $t$ to a distribution at time $t+1$.  Note that like
$\ti{\alpha}{0}$, $\ti{f}{1}$ is simply Gaussian.  On the other hand
\begin{equation}
  \label{eq:a1}
  a(1) \equiv f(1) P_{y|S} = \frac{1}{\left| \M \right|} \sum_{M \in \M}
  \prod_{i=1}^N P(y(t,M(i)|s(t,i)) \cdot \left. {\cal
      N}(F(\mu_j,\Sigma_j))\right|_{s_i}
\end{equation}
which provides an unnormalized version of the updated distribution
$\alpha(1)$ has at least two unfortunate properties:
\begin{enumerate}
\item The distribution of states is a sum of $\left| \M \right|$
  Gaussians.  If you want to track $N=100$ objects, $\alpha(1)$ will
  have $\left| \M \right| = N! \approx 10^{156}$ terms.
\item For each term in the sum, the distributions of the separate
  objects $s_j$ are independent, but that independence property does
  not hold for the sum.  So it is not true that
  \begin{equation*}
    P(s(1)|y(1)) = \prod_{j=1}^N  P\left( \left( s(1,j) \right)|y(1)
    \right).
  \end{equation*}
\end{enumerate}

Leveraging the notational clarity of \eqref{eq:a1}, I can also write
\begin{align*}
  \ti{\alpha}{T} \propto \sum_{\os{M}{t}{t=1}{T}:\ti{M}{t}\in \M}
  \prod_{i=1}^N& \left[
  P\left( \os{y}{\ti{M}{t,i}}{t=1}{T} | \os{s}{t,i}{t=1}{T}\right) \right. \\
  & \left. P\left( \os{s}{t,i}{t=1}{T}| \ti{s}{0,i} \right)
  P\left( \ti{s}{0,i} \right) \right] .
\end{align*}
Since
\begin{equation*}
  P\left( \os{y}{\ti{M}{t,i}}{t=1}{T} | \os{s}{t,i}{t=1}{T}\right) =
  \prod_{t=1}^T P\left( \ti{y}{t,\ti{M}{t,i}} | \ti{s}{t,i}\right)
\end{equation*}
and
\begin{equation*}
  P\left( \os{s}{t,i}{t=1}{T}| \ti{s}{0,i} \right) = \prod_{t=1}^T
  P\left( \ti{s}{t,i} | \ti{s}{t-1,i}\right),
\end{equation*}
\begin{equation}
  \label{eq:alphaT}
  \ti{\alpha}{T} \propto \sum_{\os{M}{t}{t=1}{T}}
  \prod_{i=1}^N  P\left( \ti{s}{0,i} \right) \prod_{t=1}^T
  P\left( \ti{y}{t,\ti{M}{t,i}} | \ti{s}{t,i}\right)
  P\left( \ti{s}{t,i} | \ti{s}{t-1,i}\right).
\end{equation}
Like \eqref{eq:a1} \eqref{eq:alphaT} is a mixture of Gaussians.  The
number of components in the mixture $\left| \ts{M}{t}{1}{T}\right| =
(N!)^T$ is absurdly large.

\subsubsection{Approximating with $\hat M$}
\label{sec:Mhat.filter}

Approximate the sum with a single $\ts{\hat M}{t}{1}{T}$.  Try to select
$\ts{\hat M}{t}{1}{T}$ to get the biggest term in the sum.
Alternatively, use a few terms.

\subsubsection{Joint Probabilistic Data Association}
\label{sec:JPDA.filter}

\subsubsection{Multiple Hypothesis Tracking}
\label{sec:MHT.filter}

\subsection{Approximation Schemes for Decoding}
\label{sec:approximation.decode}

A solution to the decoding problem
\begin{align*}
  \hat s_1^T &\equiv \argmax_{s_1^T} P(s_1^T|y_1^{t-1}) \\
            &=  \argmax_{s_1^T} P(s_1^T,y_1^{t-1})
\end{align*}
contains only a single sequence of permutations $\ts{\hat
  M}{t}{1}{T}$.  Thus I need not approximate a mixture of Gaussians.

I'll begin by considering decoding a trajectory of a single object
from a single sequence of observations.  In addition to the notation
in Section~\ref{sec:decoding}, I'll use $C(s,s',t)$ to denote the
maximum of the log of the likelihood of the data
$\ts{y}{\tau}{1}{t+1}$ over sequences $\ts{s}{\tau}{1}{t+1}$ that end
with $\ti{s}{t} = s$ and $\ti{s}{t+1} = s'$.  Assuming that $\nu(s,t)$
is quadratic, I can write the following recursion:
\begin{align}
  \nu(s,t) &= -\frac{(s-\mu_{\nu,t})^T
    \Sigma_{\nu,t}^{-1}(s-\mu_{\nu,t}) }{2} \\
  C(s,s',t) &= \nu(s,t)  -\frac{(s'-As)^T
    \Sigma_{C,t}^{-1}(s'-As) }{2} + \log(P(\ti{y}{t+1}|s') \\
  B(s,t) &= \argmax_q C(q,s,t)\\
  \nu(s,t+1) &= C(B(s,t+1),s,t)
\end{align}

\vfill \hrule
\section{Old Stuff}
\label{sec:old-stuff}

From here on I've just kept old text that I might use again as I
develop the document.\\
\hrule
\section{Model}
\label{sec:model}

The number of moving objects in my model is a constant $N_s$ over
time.  Objects may or may not be visible.  At time $t$, visibility,
position, and velocity
\begin{equation*}
  \ti{s}{j,t} \equiv \left(\ti{v}{j,t},\ti{x}{j,t},\ti{\dot x}{j,t} \right)
\end{equation*}
characterize the state of the $j\th$ object, and the vector
\begin{equation*}
  \ti{s}{t} \equiv \os{s}{j,t}{j=1}{N_s}
\end{equation*}
constitutes the entire state.  Each visibility component has three
possible values with the following interpretation
\begin{equation*}
  \ti{v}{j,t} =
  \begin{cases}
    1 & \text{Object is visible} \\
    2 & \text{Object is not visible for this frame} \\
    3 & \text{Object is not visible for at least three frames}
  \end{cases}
\end{equation*}
The components $\ti{s}{j,t}$ of the state evolve independently of each
other.  The following equations describe
$P(\ti{s}{j,t+1}|\ti{s}{j,t})$, ie, the dynamics of each component:
\begin{subequations}
  \label{eq:dynamics}
  \begin{align}
    P_{\ti{v}{j,t+1}|\ti{v}{j,t}} &= V, & V &=
    \begin{bmatrix}
      P(1 \rightarrow 1) & P(1 \rightarrow 2) & 0 \\
      P(2 \rightarrow 1) & P(2 \rightarrow 2) & P(2 \rightarrow 3) \\
      0 & P(3 \rightarrow 2) & P(3 \rightarrow 3)
    \end{bmatrix} \\
    \begin{bmatrix} \ti{x}{j,t+1} \\ \ti{\xdot}{j,t+1} \end{bmatrix}
    &= A   \begin{bmatrix} \ti{x}{j,t} \\ \ti{\xdot}{j,t} \end{bmatrix}
    + \epsilon_{j,t}, & \epsilon_{j,t} &\sim
    \normal{0}{\Sigma_{D}} \text{ iid} \\
    A &= \begin{bmatrix}
      1 & 0 & 1 & 0 \\
      0 & 1 & 0 & 1 \\
      0 & 0 & 1 & 0 \\
      0 & 0 & 0 & 1
    \end{bmatrix} &
    \Sigma_{D} &= \begin{bmatrix}
      \sigma^2_{xD} & 0 & 0 & 0 \\
      0 & \sigma^2_{xD} & 0 & 0 \\
      0 & 0 & \sigma^2_{\xdot D} & 0 \\
      0 & 0 & 0 & \sigma^2_{\xdot D}
    \end{bmatrix}
  \end{align}
\end{subequations}

An observation vector consists of $N_y$ locations
$\ti{y}{t}=\os{y}{t,i}{i=1}{N_y}$, and the probability that state $\ti{s}{t}$
would produce observation $\ti{y}{t}$ is
\begin{equation}
  \label{eq:ob}
  P(\ti{y}{t}|\ti{s}{t}) \equiv
  \begin{cases}
    0 & \text{if} N_y \neq N_{\text{visible}} \\
    \frac{1}{\left|\M \right|} \sum_{M \in \M}
    \prod_{i=1}^{N_y} P(\ti{y}{t,M(i)}|\ti{s}{t,i}) & \text{otherwise}
  \end{cases}
\end{equation}
where $\M$ is the set of permutations of $N_y$ items\footnote{This
  observation model describes indistinguishable observations.  An
  easier alternative would be objects that had different colors, or
  even easier, objects that had observable unique tags.},
\begin{equation*}
  \ti{y}{t,i}|\ti{s}{t,j} \sim
  \NormalE{\ti{x}{t,j}}{\Sigma_o}{\ti{y}{t,i}} \text{ and }
  \Sigma_o = \begin{bmatrix} \sigma_{xo}^2 & 0 \\ 0 &
    \sigma_{xo}^2 \end{bmatrix}.
\end{equation*}
In summary, aside from the initial state distribution, the model has
the following 7 degrees of freedom:
\begin{center}
  \begin{tabular}{|cp{15em}c|}
    \hline
    Symbol & Description & Degrees of freedom \\
    \hline
    $T$ & Probabilities of transition between visibility levels & 4 \\
    $\Sigma_D$ & Dynamical noise & 2 \\
    $\Sigma_o$ & Observation noise & 1 \\
    \hline
  \end{tabular} 
\end{center}

\section{Forward algorithm}
\label{sec:forward}

A complete model consists of seven parameters described in
Section~\ref{sec:model} and an initial distribution over states, ie,
$P_{s(0)}$.  Given a sequence $\os{y}{t}{t=1}{T}$ of $T$ vectors of
measurements and a complete model, the forward algorithm calculates a
sequence of \emph{forecasts} $\os{f}{t}{t=1}{T}$ and \emph{updates}
$\os{\alpha}{t}{t=1}{T}$.  Each forecast $\ti{f}{t}$ characterizes the
conditional distribution of states given all measurements up to the
previous time
\begin{equation*}
  \ti{f}{t} \rightarrow P(\ti{s}{t+1}|\os{y}{\tau}{\tau=1}{t}).
\end{equation*}
and each update $\ti{\alpha}{t}$ characterizes the conditional
distribution of states given all measurements up to the present time
\begin{equation*}
  \ti{\alpha}{t} \rightarrow P(\ti{s}{t}|\os{y}{\tau}{\tau=1}{t}).
\end{equation*}
I will denote the initial distribution over states as
$\ti{\alpha}{0}$.

The forward algorithm is recursive.  Each full iteration of the
recursion uses $\ti{y}{t+1}$ and $\ti{\alpha}{t}$ to calculate
$\ti{\alpha}{t+1}$ in the following steps:
\begin{subequations}
  \label{eq:Forward}
\begin{align}
  \label{eq:f1}
  \ti{f}{t+1} &\equiv P_{\ti{s}{t+1}|\os{y}{\tau}{\tau=1}{t}} =
  \EV{\ti{\alpha}{t}}{P(\ti{s}{t+1}|\ti{s}{t})} \\ 
  \label{eq:f2}
  \ti{a}{t+1} &\equiv
  P_{\ti{y}{t+1},\ti{s}{t+1}|\os{y}{\tau}{\tau=1}{t}} = P_{y|s}
  \ti{f}{t+1} \\
  \label{eq:f3}
  \ti{\gamma}{t+1} &\equiv P(\ti{y}{t+1}|\os{y}{\tau}{\tau=1}{t}) = \EV{\ti{f}{t+1}}{P(\ti{y}{t+1}|\ti{s}{t+1})} \\
  \label{eq:f4}
  \ti{\alpha}{t+1} &= \frac{\ti{a}{t+1}}{\ti{\gamma}{t+1}}
\end{align}
\end{subequations}
where the items on the left have the following interpretations:
\begin{description}
\item[$\ti{f}{t+1}$] A distribution of states $s$
\item[$\ti{a}{t+1}$] An unnormalized distribution of states $s$
\item[$\ti{\gamma}{t+1}$] A scalar; the conditional probability of the
  observation $\ti{y}{t+1}$ given the model and the previous observations
\item[$\ti{\alpha}{t+1}$] A distribution of states $s$
\end{description}

Each forecast and update characterizes a distribution of possible
states and thus has the same structure with the following
constituents\footnote{This is wrong.  The first update operation
  produces a mixture of Gaussians with $N_s$ components, and each
  subsequent update increases the number of components by another
  factor of $N_s$.}:
\begin{subequations}
  \label{eq:psForm}
  \begin{align}
    P_j &\equiv \begin{bmatrix} P_{v_j}(1), P_{v_j}(2), P_{v_j}(3)
    \end{bmatrix}, ~\forall j\\
    \mu_j &\equiv \begin{bmatrix} \mu_{j,1} \\ \mu_{j,2} \\ \mu_{j,3} \\
      \mu_{j,4} \end{bmatrix}, ~\forall j\\
    \Sigma_j, &~~~~~\forall j
  \end{align}
\end{subequations}

Using Eqn.~\eqref{eq:dynamics}, I implement step~\eqref{eq:f1} by:
\begin{subequations}
  \label{eq:f1I}
  \begin{align}
    P_{vfj} &= P_{v\alpha j} T\\
    \mu_{fj} &= A \mu_{\alpha j}\\
    \Sigma_{fj} &= A \Sigma_{\alpha j} A\transpose + \Sigma_D
  \end{align}
\end{subequations}

It would be nice if step~\eqref{eq:f2} lead to each component say
$a_j$ being the distribution of a weighted sum of variables with
distributions $f_k$ because that will lead to $\ti{\alpha}{t+1}$ having the
form \eqref{eq:psForm}, but $a_j$ is simply a weighted sum of the
distributions $f_k$, ie, an ugly mixture of Gaussians.

Roughly:
\begin{align*}
  a(s) &= \sum_M \prod_j P(y_{M(j)}|s_j) f(s_j) \\
  a(s_j) &= \sum_i  P(y_i|s_j) f(s_j) \sum_{M:M(j)=i}  \prod_k
  P(y_{M(k)}|s_k) f(s_k)\\
  &= \sum_i  P(y_i|s_j) f(s_j) w_{i,j} \\
  &= \sum_i  w_{i,j} \NormalE{x_j}{\Sigma_{xo}}{y_i}
  \NormalE{\mu_{fj}}{\Sigma_{fj}}{\begin{bmatrix} x_j\\ \xdot_j
    \end{bmatrix}}
\end{align*}


\subsection{Fudge}
\label{sec:fudge}

The following adjustments come to mind:
\begin{description}
\item[Hard sphere:] Require $\left|\mu_{j,1:2} - \mu_{k,1:2} \right| >
  \delta,~\forall j,k$
\item[Bounded positions:] Require $0 < \mu_{j,1} <
  \text{max}_1,~\forall j$ with a similar requirement for component 2
\item[Bounded velocities:] Require $0 < \mu_{j,3} <
  \text{max}_3,~\forall j$ with a similar requirement for component 4
\item[Bounded variance] Complicated bounds on $\Sigma_{\alpha,j}$ and
  $\Sigma_{a,j}$
\end{description}

\vfill \hrule
In \emph{svn://fraserphysics.com/ps}.
\begin{verbatim}
$Id$
\end{verbatim}

\end{document}

%%%---------------
%%% Local Variables:
%%% eval: (load-file "SeqKeys.el")
%%% eval: (TeX-PDF-mode)
%%% End:

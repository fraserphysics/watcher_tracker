\documentclass[12pt]{article}

\usepackage{amsmath,amsfonts}
\usepackage{graphicx}
\newcommand{\normal}[2]{{\cal N}(#1,#2)}
\newcommand{\NormalE}[3]{{\mathcal{N}}\left.\left(#1,#2\right)\right|_{#3}}
\newcommand{\xdot}{{\dot x}}
\renewcommand{\th}{^{\text{th}}}
\newcommand{\field}[1]{\mathbb{#1}}
\newcommand{\EV}[2]{\field{E}_{#1}\left[#2\right]}

\title{Model Based Motion Detection}
\author{Andy Fraser}

\begin{document}
\maketitle

\section{Model}
\label{sec:model}

The number of moving objects in my model is a constant $N_s$ over
time.  Objects may or may not be visible.  Visibility, position, and
velocity
\begin{equation*}
  s_j \equiv (v_j,x_j,\xdot_j)  
\end{equation*}
characterizes the state of the $j\th$ object, and the vector
\begin{equation*}
  s \equiv \left(s_1, s_2, \ldots s_{N_o} \right)
\end{equation*}
constitutes the entire state.  Each visibility component has three
possible values with the following interpretation
\begin{equation*}
  v_j =
  \begin{cases}
    1 & \text{Object is visible} \\
    2 & \text{Object is not visible for this frame} \\
    3 & \text{Object is not visible for at least three frames}
  \end{cases}
\end{equation*}
The components $s_j$ of the state $s$ evolve independently of each
other.  The following equations describe $P(s_j(t+1)|s_j(t))$, ie, the
dynamics of each component:
\begin{subequations}
  \label{eq:dynamics}
  \begin{align}
    P_{v_j(t+1)|v_j(t)} &= T, & T &=
    \begin{bmatrix}
      P(1 \rightarrow 1) & P(1 \rightarrow 2) & 0 \\
      P(2 \rightarrow 1) & P(2 \rightarrow 2) & P(2 \rightarrow 3) \\
      0 & P(3 \rightarrow 2) & P(3 \rightarrow 3)
    \end{bmatrix} \\
    \begin{bmatrix} x_j(t+1) \\ \xdot_j(t+1) \end{bmatrix}
    &= A   \begin{bmatrix} x_j(t) \\ \xdot_j(t) \end{bmatrix}
    + \epsilon_{j,t}, & \epsilon_{j,t} &\sim
    \normal{0}{\Sigma_{D}} \text{ iid} \\
    A &= \begin{bmatrix}
      1 & 0 & 1 & 0 \\
      0 & 1 & 0 & 1 \\
      0 & 0 & 1 & 0 \\
      0 & 0 & 0 & 1
    \end{bmatrix} &
    \Sigma_{D} &= \begin{bmatrix}
      \sigma^2_{xD} & 0 & 0 & 0 \\
      0 & \sigma^2_{xD} & 0 & 0 \\
      0 & 0 & \sigma^2_{\xdot D} & 0 \\
      0 & 0 & 0 & \sigma^2_{\xdot D}
    \end{bmatrix}
  \end{align}
\end{subequations}

An observation vector $y$ consists of $N_y$ locations, and the
probability that state $s$ would produce observation $y$ is
\begin{equation*}
  P(y|s) \equiv
  \begin{cases}
    0 & \text{if} N_y \neq N_{\text{visible}} \\
    \sum_{M \in \text{permutations of }N_y\text{ items }}
    \prod_{i=1}^{N_y} P(y_i|s_{M(i)}) & \text{otherwise}
  \end{cases}
\end{equation*}
where
\begin{equation*}
  y_i|s_j \sim \NormalE{x_j}{\Sigma_o}{y_i} \text{ and }
  \Sigma_o = \begin{bmatrix} \sigma_{xo}^2 & 0 \\ 0 &
    \sigma_{xo}^2 \end{bmatrix}.
\end{equation*}
In summary, the model has the following 7 degrees of freedom:
\begin{center}
  \begin{tabular}{|cp{15em}c|}
    \hline
    Symbol & Description & Degrees of freedom \\
    \hline
    $T$ & Probabilities of transition between visibility levels & 4 \\
    $\Sigma_D$ & Dynamical noise & 2 \\
    $\Sigma_o$ & Observation noise & 1 \\
    \hline
  \end{tabular} 
\end{center}

\subsection{Fudge}
\label{sec:fudge}

The following adjustments come to mind:
\begin{description}
\item[Hard sphere:] Require $\left|\mu_{j,1:2} - \mu_{k,1:2} \right| >
  \delta,~\forall j,k$
\item[Bounded positions:] Require $0 < \mu_{j,1} <
  \text{max}_1,~\forall j$ with a similar requirement for component 2
\item[Bounded velocities:] Require $0 < \mu_{j,3} <
  \text{max}_3,~\forall j$ with a similar requirement for component 4
\item[Bounded variance] Complicated bounds on $\Sigma_{\alpha,j}$ and
  $\Sigma_{a,j}$
\end{description}

\section{Forward algorithm}
\label{sec:forward}

Given a sequence $y_1^T$ of $T$ vectors of measurements $y(t):1 \leq t
\leq T$ and the the collection of  seven parameters described in
Section~\ref{sec:model}, the forward algorithm calculates a sequence
of \emph{forecasts} $f_1^T$ and \emph{updates} $\alpha_1^T$.  Each
forecast $f(t)$ characterizes the conditional distribution of states
given all measurements up to the previous time
\begin{equation*}
  f(t) \rightarrow P(s(t+1)|y_1^t),
\end{equation*}
and each update $\alpha(t)$ characterizes the conditional distribution
of states given all measurements up to the present time
\begin{equation*}
  \alpha(t) \rightarrow P(s(t)|y_1^t).
\end{equation*}

Each forecast and update characterizes a distribution of possible
states and thus has the same structure with the following
constituents\footnote{This is wrong.  The first update operation
  produces a mixture of Gaussians with $N_s$ components, and each
  subsequent update increases the number of components by another
  factor of $N_s$.}:
\begin{subequations}
  \label{eq:psForm}
  \begin{align}
    P_j &\equiv \begin{bmatrix} P_{v_j}(1), P_{v_j}(2), P_{v_j}(3)
    \end{bmatrix}, ~\forall j\\
    \mu_j &\equiv \begin{bmatrix} \mu_{j,1} \\ \mu_{j,2} \\ \mu_{j,3} \\
      \mu_{j,4} \end{bmatrix}, ~\forall j\\
    \Sigma_j, &~~~~~\forall j
  \end{align}
\end{subequations}

Each full iteration of the recursion uses $y(t+1)$ and $\alpha(t)$ to
calculate $\alpha(t+1)$ in the following steps:
\begin{align}
  \label{eq:f1}
  f(t+1) &\equiv P_{s(t+1)|y_1^t} = \EV{s(t)|y_1^t}{P(s(t+1)|s(t))} \\
  \label{eq:f2}
  a(t+1) &\equiv P_{y(t+1),s(t+1)|y_1^t} = P_{y|s} f(t+1) \\
  \label{eq:f3}
  \gamma(t+1) &\equiv P(y(t+1)|y_1^t) = \EV{f(t+1)}{P(y(t+1)|s(t+1))} \\
  \label{eq:f4}
  \alpha(t+1) &= \frac{a(t+1)}{\gamma(t+1)}
\end{align}
where the items on the left have the following interpretations:
\begin{description}
\item[$f(t+1)$] A distribution of states $s$
\item[$a(t+1)$] An unnormalized distribution of states $s$
\item[$\gamma(t+1)$] A scalar; the conditional probability of the
  observation $y(t+1)$ given the model and the previous observations
\item[$\alpha(t+1)$] A distribution of states $s$
\end{description}

Using Eqn.~\eqref{eq:dynamics}, I implement step~\eqref{eq:f1} by:
\begin{subequations}
  \label{eq:f1I}
  \begin{align}
    P_{vfj} &= P_{v\alpha j} T\\
    \mu_{fj} &= A \mu_{\alpha j}\\
    \Sigma_{fj} &= A \Sigma_{\alpha j} A^T + \Sigma_D
  \end{align}
\end{subequations}

It would be nice if step~\eqref{eq:f2} lead to each component say
$a_j$ being the distribution of a weighted sum of variables with
distributions $f_k$ because that will lead to $\alpha(t+1)$ having the
form \eqref{eq:psForm}, but $a_j$ is simply a weighted sum of the
distributions $f_k$, ie, an ugly mixture of Gaussians.

Roughly:
\begin{align*}
  a(s) &= \sum_M \prod_j P(y_{M(j)}|s_j) f(s_j) \\
  a(s_j) &= \sum_i  P(y_i|s_j) f(s_j) \sum_{M:M(j)=i}  \prod_k
  P(y_{M(k)}|s_k) f(s_k)\\
  &= \sum_i  P(y_i|s_j) f(s_j) w_{i,j} \\
  &= \sum_i  w_{i,j} \NormalE{x_j}{\Sigma_{xo}}{y_i}
  \NormalE{\mu_{fj}}{\Sigma_{fj}}{\begin{bmatrix} x_j\\ \xdot_j
    \end{bmatrix}}
\end{align*}
\end{document}

%%% Local Variables:
%%% eval: (TeX-PDF-mode)
%%% End:

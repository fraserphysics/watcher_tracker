\documentclass[12pt]{article}

\usepackage{amsmath,amsfonts}
\usepackage{graphicx}
\usepackage{showlabels}

\newcommand{\normal}[2]{{\cal N}(#1,#2)}
\newcommand{\NormalE}[3]{{\mathcal{N}}\left.\left(#1,#2\right)\right|_{#3}}
\newcommand{\xdot}{{\dot x}}
\renewcommand{\th}{^{\text{th}}}
\newcommand{\field}[1]{\mathbb{#1}}
\newcommand{\REAL}{\field{R}}
\newcommand{\RATIONAL}{\field{Q}}
\newcommand{\INTEGER}{\field{Z}}
\newcommand{\EV}[2]{\field{E}_{#1}\left[#2\right]}
\newcommand{\M}{{\cal M}}
\newcommand{\transpose}{^\top}
\newcommand{\os}[4]{{\left[ #1(#2) \right]}_{#3}^{#4}} % Object sequence
\newcommand{\ti}[2]{{#1}{(#2)}}                         % Index
%\newcommand{\ts}[4]{\os{#1}{#2}{#2=#3}{#4}} % Time series
%\newcommand{\ts}[4]{{\left[ #1(#2) \right]}_{#2=#3}^{#4}} % Sequence
\newcommand{\ts}[4]{{#1}_{#3}^{#4}} % Time series
\newcommand{\argmin}{\operatorname*{argmin}}
\newcommand{\argmax}{\operatorname*{argmax}}
\newcommand{\cS}{{\cal S}}
\newcommand{\cA}{{\cal A}}
\newcommand{\cC}{{\cal C}}
\newcommand{\logdet}{\log\left(\left|\Sigma_D\right| \left| \Sigma_O
    \right| \right)}

\title{Model Based Tracking}
\author{Andy Fraser}

\begin{document}
\maketitle

\section*{Introduction}
\label{sec:introduction}

For tracking, I want the maximum a posteriori (MAP) or \emph{decoded}
track ie,
\begin{align*}
  \ts{\hat s}{\tau}{1}{T} &\equiv \argmax_{\ts{s}{\tau}{1}{T}}
  P(\ts{s}{\tau}{1}{T}|\ts{y}{\tau}{1}{T}) \\
  &= \argmax_{\ts{s}{\tau}{1}{T}} P(\ts{s}{\tau}{1}{T},\ts{y}{\tau}{1}{T}).
\end{align*}

Note that the MAP track is different than the sequence of
\emph{filtered} forecasts.  A filtered estimate, ie,
\begin{equation*}
  P(S(t)|\ts{y}{\tau}{1}{t}),
\end{equation*}
is the entire a posteriori distribution at each time given the
observations up to that time.  When I began thinking about tracking
and writing this document, I thought that the best approach would be
to do filtering.  I've left some of the work I did while thinking
about filtering in Appendix~\ref{sec:filtering} which you may safely
ignore.

\subsection*{Model Assumptions}
\label{sec:model-assumptions}

The following assumptions describe a \emph{state space model}:
\begin{description}
\item[The state dynamics are Markov:]
  \begin{equation}
    \label{eq:Markov}
    P(\ts{s}{\tau}{1}{t-1},\ts{s}{\tau}{t+1}{T}|\ti{s}{t}) =
    P(\ts{s}{\tau}{1}{t-1}|\ti{s}{t}) ~ P(\ts{s}{\tau}{t+1}{T}|\ti{s}{t}),
  \end{equation}
  ie, the future is conditionally independent of the past given the present.
\item[The observations depend only on the states:] Or more precisely,
  the observation at time $t$ is conditionally independent of
  everything else given the state at time $t$, ie,
  \begin{equation}
    \label{eq:IndependentY}
    P(\ti{y}{t},z|\ti{s}{t}) =  P(\ti{y}{t}|\ti{s}{t}) ~  P(z|\ti{s}{t})
  \end{equation}
  where $z$ can be any collection of other states and observations.
\end{description}
If these assumptions hold, the following functions completely describe
a model:
\begin{description}
\item[State transition probability:] $P(\ti{s}{t+1}|\ti{s}{t})$
\item[Conditional observation probability:] $P(\ti{y}{t}|\ti{s}{t})$
\item[Initial state distribution] $P(\ti{s}{1})$
\end{description}

\subsection*{Decoding}
\label{sec:decoding}

I will use the following definitions to describe decoding:
\begin{align*}
  u(\ts{s}{\tau}{1}{t}) & \quad \text{Utility of state sequence }
  \ts{s}{\tau}{1}{t}\\
  & \quad \equiv \log \left( P(\ts{y}{\tau}{1}{t},\ts{s}{\tau}{1}{t}) \right)
  \\
  \nu(s,t) & \quad \text{Utility of best sequence ending with }
  \ti{s}{t} = s \\ 
  &  \quad \equiv \max_{\ts{s}{\tau}{1}{t}:\ti{s}{t}=s} u(\ts{s}{\tau}{1}{t}) \\
  \omega(s,s',t) & \quad \text{Utility of best sequence ending with }
  \ti{s}{t},\ti{s}{t+1} = s,s' \\
  &  \quad \equiv \max_{\ts{s}{\tau}{1}{t+1}:\ti{s}{t}=s \&\ti{s}{t+1}=s'}
  u(\ts{s}{\tau}{1}{t+1}) \\
  B(s',t) & \quad \text{Best predecessor state given } \ti{s}{t+1}=s'\\
  & \quad \equiv \argmax_{s} \omega(s,s',t))
\end{align*}
I can evaluate the functions $B(*,t)$ and $\nu(*,t)$ recursively as
follows:
\begin{align*}
  \omega(s,s',t) &= \nu(s,t) + \log\left( P_{\ti{s}{t+1}|\ti{s}{t}}(s'|s) \right) +
  \log\left( P_{\ti{y}{t+1}|\ti{s}{t+1}}(\ti{y}{t+1}|s') \right) \\
  \nu(s,t+1) &= \omega(B(s,t),s,t)
\end{align*}
Given the functions $B(*,t)$ and $\nu(*,t)$ $\forall
t\in[1,\ldots,T]$, the following procedure decodes the best sequence
of states $ \ts{\hat s}{\tau}{1}{T}$ from a sequence of observations $
\ts{y}{\tau}{1}{T}$
\begin{align*}
  {\ti{{\hat s}}{T}} &= \argmax_s \nu(s,T) \\
  & \text{for } t \text{ from } T-1 \text{ to } 1: \\
  & \quad \ti{\hat s}{t} = B( \ti{\hat s}{t+1},t)
\end{align*}

\section{Model Version One}
\label{sec:model1}

In this section I describe the basic model \emph{MVI}.  I'll develop
algorithms and write code for \emph{MVI} for the baseline prototype of
the watcher project.  Later, I will compare proposed enhancements
against the performance of \emph{MVI}.

The number of moving objects is a constant $N$ over time.  At each
time $t$, the separate objects $\ti{s}{t,j}$ that constitute the state
evolve independently of each other.  The following equations describe
$P(\ti{s}{t+1,j}|\ti{s}{t,j})$, ie, the dynamics of each object:
\begin{subequations}
  \label{eq:dynamics}
  \begin{align}
    \ti{s}{t+1,j} &= A  \ti{s}{t,j} + \ti{\epsilon}{t,j}, &
    \ti{\epsilon}{t,j} &\sim \normal{0}{\Sigma_{D}} \text{ iid} \\
    A &= \begin{bmatrix}
      1 & 0 & \tau & 0 \\
      0 & 1 & 0 & \tau \\
      0 & 0 & 1 & 0 \\
      0 & 0 & 0 & 1
    \end{bmatrix} &
    \Sigma_{D} &= \begin{bmatrix}
      \sigma^2_{xD} & 0 & 0 & 0 \\
      0 & \sigma^2_{xD} & 0 & 0 \\
      0 & 0 & \sigma^2_{\xdot D} & 0 \\
      0 & 0 & 0 & \sigma^2_{\xdot D}
    \end{bmatrix}
  \end{align}
\end{subequations}
Each object has the following components at time $t$:
\begin{align*}
  &&s_{0}(t,j) & \text{ horizontal position} \\
  &&s_{1}(t,j) & \text{ vertical position} \\
  &&s_{2}(t,j) & \text{ horizontal velocity} \\
  &&s_{3}(t,j) & \text{ vertical velocity}
\end{align*}
In addition to the moving objects, a complete description of a state
$\ti{s}{t}$ includes a map $\ti{M}{t}\in \M$, where $\M$ is the set of
permutations of $N$ items.  I assume that $\ti{M}{t}$ is distributed
uniformly and independently of everything else, ie,
\begin{equation*}
  P(\ti{M}{t}) = \frac{1}{\left| \M\right|} =  \frac{1}{N!}.
\end{equation*}

An observation vector consists of locations
$\ti{y}{t}=\os{y}{t,i}{i=1}{N}$.  The probability that state
$\ti{s}{t}$ would produce observation $\ti{y}{t}$ is\footnote{This
  observation model describes indistinguishable observations.  An
  easier alternative would be objects that had different colors, or
  even easier, objects that had observable unique tags.}
\begin{equation}
  \label{eq:ob}
  P(\ti{y}{t}|\ti{s}{t}) =
    \frac{1}{\left|\M \right|} \sum_{M \in \M}
    \prod_{i=1}^{N} P(\ti{y}{t,M(i)}|\ti{s}{t,i}).
\end{equation}
with normal conditional observation distributions
\begin{equation*}
  \ti{y}{t,j}|\ti{s}{t,i} \sim
    \NormalE{O\ti{s}{t,i}}{\Sigma_O}{\ti{y}{t,j}}
\end{equation*}
where each mean comes from the projection
\begin{equation*}
  O = \begin{bmatrix} 1 & 0 & 0 & 0\\
    0 & 1 & 0 & 0 \end{bmatrix}
\end{equation*}
and each covariance is
\begin{equation*}
  \Sigma_O = \begin{bmatrix} \sigma_{xO}^2 & 0 \\ 0 &
    \sigma_{xO}^2 \end{bmatrix}.
\end{equation*}
In summary, aside from the initial state distribution, the model has
the following 3 degrees of freedom:
\begin{center}
  \begin{tabular}{|cp{15em}c|}
    \hline
    Symbol & Description & Degrees of freedom \\
    \hline
    $\Sigma_D$ & Dynamical noise & 2 \\
    $\Sigma_O$ & Observation noise & 1 \\
    \hline
  \end{tabular} 
\end{center}

\subsection{Decoding}
\label{sec:decode1}

A solution to the decoding problem
\begin{align*}
  \ts{\hat s}{\tau}{1}{T} &\equiv \argmax_{\ts{s}{\tau}{1}{T}}
  P(\ts{s}{\tau}{1}{T}|\ts{y}{\tau}{1}{T}) \\
            &=  \argmax_{\ts{s}{\tau}{1}{T}} P(\ts{s}{\tau}{1}{T},\ts{y}{\tau}{1}{T})
\end{align*}
contains only a single sequence of permutations $\ts{\hat
  M}{t}{1}{T}$.  Thus I need not approximate a mixture of Gaussians.

\subsubsection{Single Sequence}
\label{sec:single-sequence}

I'll begin by considering decoding a trajectory of a single object
from a single sequence of observations.  Using the notation in the
introduction, and assuming that $\nu(s,t)$ is quadratic, I can write
the following recursion:
\begin{align}
  \nu(s,t) &= -\frac{1}{2}(s-\mu_{t})^T
    \Sigma_{t}^{-1}(s-\mu_{t}) + R_t \nonumber \\
  \label{eq:decode_u'}
  \omega(s,s',t) &= \nu(s,t) -\frac{1}{2} \logdet  - 
  \frac{1}{2}(s'-As)^T \Sigma_{D}^{-1} (s'-As)  \nonumber \\
  &\quad - \frac{1}{2}(\ti{y}{t+1} - O s')^T \Sigma_{O}^{-1}(\ti{y}{t+1}
    - O s') \\
  B(s,t) &= \argmax_{q} \omega(q,s,t) \nonumber \\
  \label{eq:decode_B}
  &= \left( \Sigma_t^{-1} + A^T \Sigma_D^{-1} A \right)^{-1} \left(
    \Sigma_t^{-1} \mu_t + A^T \Sigma_D^{-1} s \right) \\
  \nu(s,t+1) &= \omega(B(s,t),s,t) \nonumber \\
  &\equiv  -\frac{1}{2}(s-\mu_{t+1})^T
  \Sigma_{t+1}^{-1}(s-\mu_{t+1}) + R_{t+1} \nonumber \\
  \label{eq:new_Sigma}
  \Sigma_{t+1}^{-1} & = O^T\Sigma_O^{-1} O + \left( \Sigma_D + A \Sigma_t
    A^T \right)^{-1} \\
  \label{eq:new_mu}
  \mu_{t+1} &= A \mu_t + \Sigma_{t+1} O^T \Sigma_O^{-1} \Delta_y \\
  \label{eq:new_R}
  R_{t+1} &= R_t -\frac{1}{2} \left( \Delta_y^T \Sigma_y^{-1} \Delta_y
    + \logdet \right),
\end{align}
where
\begin{align}
  \label{def:Delta_y}
  \Delta_y &\equiv \ti{y}{t+1} - OA \mu_t \\
  \label{def:Sigma_y}
  \Sigma_y &\equiv O(A\Sigma_t A^T + \Sigma_D)O^T + \Sigma_O \\
  \label{eq:Sigma_yI}
  \Sigma_y^{-1} &= \Sigma_O^{-1} - \Sigma_O^{-1} O \Sigma_{t+1} O^T
  \Sigma_O^{-1}.
\end{align}
While Defs.~\eqref{def:Delta_y} and  \eqref{def:Sigma_y} are simply
abbreviations, Eqn.~\eqref{eq:Sigma_yI} requires effort to derive.

I derive \eqref{eq:decode_B} by
solving\footnote{The derivation of \eqref{eq:decode_B}:
  \begin{align*}
    \frac{d \omega(q,s,t)}{d q} &= -\Sigma_t^{-1}(q-\mu_t) + A^T \Sigma_D^{-1}
    (s - Aq) \\
    &= \Sigma_t^{-1}\mu_t + A^T \Sigma_D^{-1} s -(\Sigma_t^{-1} +
    A^T\Sigma_D^{-1}A)q \\
    q &= \left( \Sigma_t^{-1} + A^T\Sigma_D^{-1}A\right)^{-1} \left(
      \Sigma_t^{-1}\mu_t + A^T \Sigma_D^{-1} s \right)
  \end{align*}
}
\begin{equation*}
  \frac{d \omega(q,s,t)}{d q} = 0.
\end{equation*}
Note that the independent variable $s$ in \eqref{eq:decode_B} is the
state at the future time $t+1$, while in \eqref{eq:decode_u'}, $s$ is
the state at the earlier time $t$.  See Appendix~\ref{app:decode} for
the derivation of \eqref{eq:new_Sigma}, \eqref{eq:new_mu}, and
\eqref{eq:new_R}.  I am surprised that the resulting recursion for
$\nu(s,t)$ is exactly Kalman filtering.

For efficiency, use the following procedure to implement the forward
recursion:
\begin{description}
\item[Calculate the state forecast mean and covariance:]
  \begin{align*}
    \mu_a &= A\mu_t \\
    \Sigma_a &= A \Sigma_t A^T + \Sigma_D
  \end{align*}
\item[Calculate the inverse covariance of the forecast observation:]
  \begin{equation*}
   \Sigma_y^{-1} = \left( O\Sigma_a O^T + \Sigma_O \right)^{-1}
  \end{equation*}
\item[Calculate the Kalman gain matrix:]
  \begin{equation*}
    K = \Sigma_a O^T \Sigma_y^{-1}
  \end{equation*}
\item[Calculate the forecast error:]
  \begin{equation*}
    \Delta_y = \ti{y}{t+1} - O\mu_a
  \end{equation*}
\item[Calculate the updated mean and covariance:]
  \begin{align}
    \label{eq:Sigma_alg}
    \mu_{t+1} &= \mu_a +  K\Delta_y \\
    \label{eq:mu_alg}
    \Sigma_{t+1} &= (1-KO)\Sigma_a
  \end{align}
\item[Calculate the new $R$:]
  \begin{equation}
    \label{eq:R_alg}
    R_{t+1} = R_t - \frac{1}{2} \left(\Delta_y^T \Sigma_y^{-1}
      \Delta_y + \logdet \right)
  \end{equation}
\end{description}

\subsubsection{Multiple Objects}
\label{sec:multiple}

Consider the possible permutations at each time to be a discrete
component of the state.  Thus
\begin{equation*}
  \ti{s}{t} \in {\cal S} \equiv {\cal M} \times {\cal X}^{N},
\end{equation*}
where $N$ is the number of objects or targets, $\cal M$ is the set of
permutations of $N$ objects, and ${\cal X}$ is the state space of a
single object.  I denote the components of a particular state as
follows:
\newcommand{\bx}{{\mathbf{x}}}
\newcommand{\by}{{\mathbf{y}}}
\begin{align*}
  s &\equiv (M,\bx) \\
  M & \text{ is a permutation} \\
  \bx &\equiv (x_1,x_2,\ldots,x_N).
\end{align*}
The independence assumptions permit the following decomposition:
\begin{align}
  \mathbf{P} &\equiv
  P(\ts{\by}{\tau}{1}{t},\ts{\bx}{\tau}{1}{t},\ts{M}{\tau}{1}{t})
  \nonumber \\
  &= P(\ts{M}{\tau}{1}{t})
  P(\ts{\by}{\tau}{1}{t},\ts{\bx}{\tau}{1}{t}|\ts{M}{\tau}{1}{t})
  \nonumber \\
  &= \frac{1}{(N!)^t} \prod_{\tau=1}^t \left(
    P(\ti{\by}{\tau}|\ti{\bx}{\tau},\ti{M}{\tau})
    P(\ti{\bx}{\tau}|\ti{\bx}{\tau-1}) \right) \nonumber \\
  \label{eq:factors}
  &= \frac{1}{(N!)^t} \prod_{\tau=1}^t  \prod_{k=1}^N \left(
  P(\ti{y_{M_k(\tau)}}{\tau}|\ti{x_k}{\tau})
  P(\ti{x_k}{\tau}|\ti{x_k}{\tau-1}) \right).
\end{align}
\newcommand{\mmm}{\max_{\ts{M}{\tau}{1}{t}:\ti{M}{t}=M}}
\newcommand{\mxx}{\max_{\ts{\bx}{\tau}{1}{t}:\ti{\bx}{t}=\bx}} Now I
use Eqn.~\eqref{eq:factors} to define $\nu_k(x_k,t|M))$, which is the
contribution to the total utility by the $k^{\text{th}}$ target and
its observations given the best sequence of $M$s that ends in $M$, as
follows:
\begin{align*}
  \nu(M,\bx,t) &= \mmm \quad \mxx \quad \log(\mathbf{P})\\
  &= t\log(N!) + \mmm \quad \mxx \quad \sum_k \sum_\tau \Big(\\
  & \quad \quad \log(P(\ti{y_{M_k(\tau)}}{\tau}|\ti{x_k}{\tau})) +
  \log( P(\ti{x_k}{\tau}|\ti{x_k}{\tau-1}) ) \Big) \\
  & \equiv t\log(N!) + \sum_k \nu_k(x_k,t|M).
\end{align*}
Letting $\mu(k,t|M)$, $\Sigma^{-1}(k,t|M)$ and $R(k,t|M)$ denote the
parameters that characterize $\nu_k(x_k,t|M)$, and recognizing that
$R(k,t|M) = \nu_k(\mu(k,t|M),t|M)$, I write
\begin{align*}
  \nu(M,\bx,t) &= t\log(N!) + \sum_k \Big( R(k,t|M) \\
  &\quad - \frac{1}{2} \left( x_k-\mu(k,t|M) \right)^T
  \Sigma^{-1}(k,t|M) \left( x_k-\mu(k,t|M) \right) \Big).
\end{align*}
Each $B(s,t)$ maps from $(M',\bx')$ at time $t+1$ to the best pair
$(M,\bx)$ at time $t$, and each $\nu(s,t)$ maps from a pair $(M,\bx)$
at time $t$ to the utility of the best path ending at $(M,\bx,t)$.
For each $M$, I use the values of $\sum_k R(k,t|M)$ to determine the
best predecessor permutation, and I drop the term $\log(N!)$ because
it is independent of both $M$ and $\bx$.

I have implemented an algorithm around the following recursion step:
\begin{verse}
  At time $t$ given $(\mu(k,t|M), \Sigma(k,t|M), R(k,t|M))~\forall
  (M,k)$:\\
  For each possible permutation $M'$ at time $t+1$:\\
  \hspace{2em} For each possible predecessor $M$ at time $t$:\\
  \hspace{4em} Calculate $\omega(M,M',t) = \sum_k R(k,t+1|M,M')$ \\
  \hspace{2em} Set $B_M(M',t+1) = \tilde M = \argmax_{M}\omega(M,M',t)$\\
  \hspace{2em} For each object $k$:\\
  \hspace{4em} Calculate $(\mu(k,t+1|M'), \Sigma(k,t+1|M'), R(k,t+1|M'))$\\
  \hspace{4em} from Eqns.~\eqref{eq:Sigma_alg}--\eqref{eq:R_alg} using
  the initial values\\
  \hspace{4em} $(\mu(k,t|\tilde M), \Sigma(k,t|\tilde M),R(k,t|\tilde
  M))$
\end{verse}
(Note: The calculation of $\omega$ on the fourth line is incorrect
and will be revised in the next version.)

\section{Model Version Two: Variable Visibility}
\label{sec:model2}

In MVI of the previous section, I described the state at each time as
a combination of the positions and velocities of the targets and the
map from targets to observations.  Recall from
Section~\ref{sec:multiple}:
\begin{align*}
  s &\equiv (M,\bx) \\
  M & \text{ is a permutation} \\
  \bx &\equiv (x_1,x_2,\ldots,x_N).
\end{align*}
For \emph{MVII} I introduce a discrete \emph{visibility} for each
target, ie, \newcommand{\bv}{{\mathbf{v}}}
\begin{align*}
  s &\equiv (M,\bx,\bv) \\
  \bv &\equiv (v_1,v_2,\ldots,v_N).
\end{align*}
I assume that only targets with $v_k=0$ are visible and that
visibility ($v$), phase space dynamics ($x$), and association ($M$) are
independent.

The following pair of equations define the model:
\begin{equation}
  P(\ti{s}{t+1}|\ti{s}{t}) = \frac{1}{\left| \M \right|} \prod_i
  P(\ti{x_i}{t+1}|\ti{x_i}{t})\, P(\ti{v_i}{t+1}|\ti{v_i}{t})
\end{equation}
\begin{equation}
  P(y|s) =
  \begin{cases}
    0 & \text{If } \left| \{i:v_i=0 \} \right| \neq \text{number of
      observations} \\
    \prod_{i:v_i=0} P(y(M(i))|x_i) & \text{Otherwise}
  \end{cases}
\end{equation}
The only new parameters are $P(\ti{v_i}{t+1}|\ti{v_i}{t})$ the
probabilities of transitions between visibility states.  I find the
modifications to algorithms for MVI required for MVII by observing
\begin{align}
  \omega(M,\bx,\bv,M',\bx',\bv't) &= \nu(M,\bx,\bv) - \log(N!) + \log \left(
    P(\bx'|\bx)\right) \nonumber \\
  & \quad + \log \left( P(\bv'|\bv)\right) + \log \left(
  P(\ti{y}{t+1}|M',\bx')\right)
\end{align}
and noting that $\log \left( P(\bv'|\bv)\right)$ is the only new term
and
\begin{equation}
  \label{eq:1}
   \log \left( P(\bv'|\bv)\right) = \sum_i \log \left(
     P(v_i'|v_i)\right),
\end{equation}
where the subscript $i$ identifies the target.

Doing a forward iteration on a target that is not visible requires
expressions like Eqns.~\eqref{eq:decode_u'}-\eqref{eq:Sigma_yI} but
without an observation, ie,

\begin{align}
  \nu(s,t) &= -\frac{1}{2}(s-\mu_{t})^T
    \Sigma_{t}^{-1}(s-\mu_{t}) + R_t \nonumber \\
  \omega(s,s',t) &= \nu(s,t) -\frac{1}{2} \log(\left|\Sigma_D\right|)  - 
  \frac{1}{2}(s'-As)^T \Sigma_{D}^{-1} (s'-As)  \nonumber \\
  B(s,t) &= \argmax_{q} \omega(q,s,t) \nonumber \\
  &= \left( \Sigma_t^{-1} + A^T \Sigma_D^{-1} A \right)^{-1} \left(
    \Sigma_t^{-1} \mu_t + A^T \Sigma_D^{-1} s \right) \nonumber \\
  \nu(s,t+1) &= \omega(B(s,t),s,t) \nonumber \\
  &\equiv -\frac{1}{2}(s-\mu_{t+1})^T
  \Sigma_{t+1}^{-1}(s-\mu_{t+1}) + R_{t+1} \nonumber \\
  \label{eq:new_Sigma_noy}
  \Sigma_{t+1} & = \left( \Sigma_D + A \Sigma_tA^T \right)^{-1} \\
  \label{eq:new_mu_noy}
  \mu_{t+1} &= A \mu_t \\
  \label{eq:new_R_noy}
  R_{t+1} &= R_t -\frac{1}{2}  \log(\left|\Sigma_D\right|).
\end{align}

\section{Model Version Three: Background False Alarms}
\label{sec:model3}

This model, \emph{MVIII}, generates hits like MVII except that in
addition it generates noise hits or \emph{false alarms}.  The false
alarms are drawn from the initial $y$ distribution
$\normal{\mu_0}{\Sigma_0}$ and the number of hits is drawn from a
Poisson distribution with expected value $\lambda$.  The number of
false alarms, $N_{\text{FA}}$, is a characteristic of a state, ie, $s
\equiv (M,\bx,\bv,N_{\text{FA}})$ and
\begin{equation}
  \label{eq:2}
  P(\ti{s}{t+1}|\ti{s}{t},\ti{M}{t+1},\ti{\bx}{t+1}) =
  \frac{e^{-\lambda}\lambda^{N_{\text{FA}}}}{N_{\text{FA}}!}.
\end{equation}
And the probability density of $y$ given that it is a false alarm is
\begin{equation*}
  P(y|\text{FA}) = \NormalE{\mu_0}{\Sigma_0}{y}.
\end{equation*}
For decoding, if you attribute the observations $y_1^{N_{\text{FA}}}$
to false alarms at time $t$, add\footnote{See Appendix
  \ref{sec:counting}} $\sum_k^{N_{\text{FA}}} \log(P(y_k|\text{FA}))$
and $\log \left( e^{-\lambda} \lambda^{N_{\text{FA}}} \right)$ to the
value of $\nu$.\marginpar{Are you sure?}

\section{Model Version Four: Variable Number of Targets}
\label{sec:model4}

\emph{MVIV} creates $N_{\text{new}}$ new targets at each time step.
$N_{\text{new}}$ has a Poisson distribution with a small mean; at most
times there are no new states.  The positions and velocities of the
new states have the same distribution as initial states.  I assume
that new states are visible when they arise to avoid carrying many
invisible states for decoding.

In the decoding algorithm, the cost of attributing the observations
$y_1^{N_{\text{new}}}$ to new targets at time $t$ is the
addition\footnote{See Appendix \ref{sec:counting}} of
$\sum_k^{N_{\text{new}}} \log(P(y_k|\text{new}))$ and $\log \left(
  e^{-\lambda} \lambda^{N_{\text{new}}}\right)$ to the value of
$\nu$.\marginpar{Are you sure?}

While it would be simpler to treat false alarms and new targets with
the same model mechanism, since new targets will frequently arise at
edges while false alarms will happen in other areas more frequently, I
expect better performance by using separate mechanisms to model the
two kinds of events.

\textbf{Caution:} Since the dimension of the state changes with time,
we should make sure that the results of algorithms do not depend on
coordinates.

Ultimately MVIV will have the following properties:
\begin{itemize}
\item Start with a random number of targets
\item New targets created randomly
\item Old targets removed after being invisible for some number of
  frames
\end{itemize}

\section{Algorithms and Approximations}
\label{sec:algorithms}

For MVIV, a state $s \equiv (M, \bx, \bv, N_{\text{FA}},
N_{\text{targets}})$ has the following components:
\begin{description}
\item[$M$:] A map from a subset of the targets to a subset of the observations
\item[$\bx$:] The positions of the targets
\item[$\bv$:] The \emph{visibility} of each of the targets
\item[$N_{\text{FA}}$:] The number of false alarms (derivable from $M$)
\item[$N_{\text{targets}}$:] The number of targets (derivable from $\bx$)
\end{description}
I divide the components in two parts: The \emph{target} part
$(\bx,\bv)$; and the \emph{association} part, $\cA \equiv (M,
N_{\text{FA}}, N_{\text{targets}})$.  For each time $t$ in the forward
pass of the Viterbi algorithm, I need $\nu(s,t)$ and $B(s,t)$.  I keep
track of the association parts of states, treating them as discrete
states in the Viterbi algorithm.  At time $t$ each $\cA$ has: An
explanation for each observation; a pointer to the best predecessor,
$B(\cA,t)$, which inductively also explains all previous observations;
and a utility $\nu(\cA,t) = \max_{\bx,\bv} \nu(s,t)$.  To make the $B$
and $\nu$ functions complete, I keep track of a mean $\mu_k(t)$ and a
covariance $\Sigma_k(t)$ for each target $(\bx_k,\bv_k)$ used as part
of an explanation in each $\cA$ or any of the predecessors (See
Eqns.~\eqref{eq:mu_alg} and \eqref{eq:Sigma_alg}).

\subsection{Clusters}
\label{sec:clusters}

I believe that if, over an interval of time, the observations are
clustered into groups that are far apart, I can reduce the number of
computations in the forward part of the Viterbi algorithm by breaking
the problem into pieces that match the clusters in the data.

I define a cluster of associations as a set associations each of which
along with its predecessors explains the same set of observations.

\subsubsection{Merging}
\label{sec:merging}

To merge two clusters, $\cC_1$ and $\cC_2$, create a new association
from each pair drawn from the two clusters, ie,
\begin{align}
   \cC_3 &= \text{Merge}(\cC_1,\cC_2) \\
  \forall \cA_{i}, \cA_{j} \text{ such that } \cA_{i} \in \cC_1 &\text{
    and } \cA_{j} \in \cC_2 ~~ \cA_k = \text{Merge}(\cA_{i}, \cA_{j})
  \in \cC_3 \\
  \text{Merge}(\cA_{i}, \cA_{j}): \\
  M_k &= \text{concatenate } M_i, M_j \\
  N_{\text{FA},k} &= N_{\text{FA},i} + N_{\text{FA},j} \\
  N_{\text{targets},k} &= N_{\text{targets},i} + N_{\text{targets},j}
  \\
  \nu_k &= \nu_i +\nu_j
\end{align}

\subsubsection{Branching}
\label{sec:branching}

Branching, $(\cC_1,\cC_2) = \text{Branch}(\cC_3)$ seems harder.  In
particular,
\begin{description}
\item[Q:] Can one break each association in a parent
  cluster into fragments and then group those association fragments to
  form child clusters?
\item[A:] No.  All of the fragments that get joined to make a new
  cluster must explain the same present and past observation
  components, but of the fragments that explain a particular subset of
  the present observations, some pairs will explain different subsets
  of past observations.
\item[Q:] How should I deal with inconsistent fragments?
\item[A:] Find the parent association with the highest utility and use
  its fragments to seed the child clusters.  For parent associations
  with lower utility, keep only those fragments that are consistent
  with (explain the same past observation components) as the seed
  fragments.
\item[Q:] How should $\nu$ be apportioned between the fragments?
\item[A:] $\nu$ should be tied to the explanation of past observation
  components.  While it is clear how to handle targets, there is more
  than one way to handle false alarms.  I try to determine the minimum
  set of historical observations that is consistent with the set of
  association fragments that is best in the sense that I retain the
  possible fragments with highest utility.  Then I calculate the
  utility for each fragment by its explanation of that set of
  historical observations.  I discard parent associations that are
  inconsistent with the chosen fragmenting of historical observations.
  I discard historical observations that all retained fragments agree
  were false alarms.  And I archive \emph{dead targets} in the best
  fragment that are expendable in the sense that none of the
  associated observations are connected with live targets in any of
  the other retained fragments.
\end{description}
First branch the association with the highest score, then discard all
associations whose branchings are inconsistent with the first.  A
branching is inconsistent if the targets going into the branches do
not explain the same subsets of observations from the first time to
the present.  Since branching might discard an association that would
prove to be superior on the basis of future observations, one should
flag for review such discardings as possible errors.

\subsubsection{Forward Step}
\label{sec:forward_b_m}

Having processed observations $\ts{y}{\tau}{1}{t-1}$ implement the
following steps to process the observations $\ti{y}{t}$
\begin{enumerate}
\item Find clusters of targets $\ti{x}{t-1}$ defined by their children
  $\ti{x}{t}$ explaining clusters of observations $\ti{y}{t}$
  \begin{description}
  \item[Q:] What about invisible targets?
  \item[A:] Cluster membership for invisible targets is determined by
    children.  If a target has no visible children, then it is in a
    cluster by itself.
  \end{description}
\item Reform clusters of associations based on step 1
\item Propagate the clusters
\end{enumerate}

\appendix
\section{Counting}
\label{sec:counting}

I am confused by counting problems in the models.
\begin{description}
\item[Q:] Why in Section \ref{sec:model4} do I add the term $\log
  \left( e^{-\lambda} \lambda^{N_{\text{new}}} \right)$ instead of
  $\log \left(
    \frac{e^{-\lambda}\lambda^{N_{\text{new}}}}{N_{\text{new}}!}\right)$?
  (There is a similar term in Section \ref{sec:model3} for
  $N_{\text{FA}}$.)
\item[A:] There are $N_{\text{new}}!$ different, but operationally
  equivalent, ways to assign $N_{\text{new}}$ targets to
  $N_{\text{new}}$ locations.
\item[Q:] Suppose that at time $t=0$ targets sit at locations
  $\ti{x_a}{0}$ and $\ti{x_b}{0}$ and that at time $t=1$ targets sit
  at locations $\ti{x_c}{0}$ and $\ti{x_d}{0}$.  If I say that
  $\ti{s}{0}\equiv \left(x_a,x_b \right)$ and $\ti{s}{1}\equiv
  \left(x_c,x_d \right)$, should I write
  \begin{equation}
    \label{eq:3}
    P(\ti{s}{1}|\ti{s}{0}) =  P(x_c|x_a) \,  P(x_d|x_b) +  P(x_c|x_b)
    \,  P(x_d|x_a)
  \end{equation}
  or simply
  \begin{equation}
    \label{eq:4}
    P(\ti{s}{1}|\ti{s}{0}) =  P(x_c|x_a) \,  P(x_d|x_b)
  \end{equation}
\item[A:] Use \eqref{eq:4}.  The second term in \eqref{eq:3} is part
  of a different sequence of state transitions.
\end{description}

\section{Derivation of Formulas for Decoding the Trajectory of a Single Object}
\label{app:decode}

I obtain \eqref{eq:new_Sigma} and \eqref{eq:new_mu} by expanding
$\omega(B(s,t),s,t)$ and completing the square.  I'll abbreviate $B(s,t)$
with $B$, using the notation (see Eqn.~\eqref{eq:decode_B})
\begin{align*}
  B &\equiv \left( \Sigma_t^{-1} + A^T \Sigma_D^{-1} A \right)^{-1}
  \left( \Sigma_t^{-1} \mu_t + A^T \Sigma_D^{-1} s \right) \\
  &= G + Fs \\
  G &\equiv \left( \Sigma_t^{-1} + A^T \Sigma_D^{-1} A \right)^{-1}
  \Sigma_t^{-1} \mu_t \\
  F &\equiv  \left( \Sigma_t^{-1} + A^T \Sigma_D^{-1} A \right)^{-1}
  A^T \Sigma_D^{-1}.
\end{align*}
I find (see Eqn.~\eqref{eq:decode_u'})
\begin{align*}
  \nu(s,t+1) &= -\frac{1}{2}(B-\mu_{t})^T \Sigma_{t}^{-1}
  (B-\mu_{t}) + R_t\\
  &\quad - \frac{1}{2} (s-AB)^T \Sigma_{D}^{-1} (s-AB)\\
  &\quad - \frac{1}{2}(\ti{y}{t+1}-O s)^T \Sigma_{O}^{-1}
  (\ti{y}{t+1}-Os) -\frac{1}{2} \logdet\\
  -2 \nu(s,t+1) &= (G-\mu_{t}+Fs)^T \Sigma_{t}^{-1} (G-\mu_{t}+Fs) \\
  &\quad + ((1-AF)s-AG)^T \Sigma_{D}^{-1} ((1-AF)s-AG)\\
  &\quad + (\ti{y}{t+1}-O s)^T \Sigma_{O}^{-1} (\ti{y}{t+1}-Os) -2R_t + \logdet \\
  &\equiv s^T q s - 2s^T l + c.
\end{align*}
To express $\nu_{t+1}$ in terms of $\Sigma_{t+1}$, $\mu_{t+1}$, and
$R_{t+1}$, I will use the following formulas:
\begin{align*}
  \Sigma_{t+1}^{-1} &= q \\
  \mu_{t+1} &= \Sigma_{t+1} l \\
  R_{t+1} &= -\frac{1}{2} \left( c - \mu_{t+1}^T \Sigma_{t+1}^{-1}
    \mu_{t+1} \right) .
\end{align*}
The quadratic term is
\begin{align*}
  q &= F^T\Sigma_t^{-1}F + (1-AF)^T\Sigma_D^{-1}(1-AF) +
  O^T\Sigma_O^{-1}O \\
  &= O^T\Sigma_O^{-1}O + \Sigma_D^{-1} - \Sigma_D^{-1}A
  ( \Sigma_t^{-1} + A^T\Sigma_D^{-1}A )^{-1}A^T\Sigma_D^{-1}\\
  &= O^T\Sigma_O^{-1}O + (\Sigma_D + A \Sigma_t A^T)^{-1}
\end{align*}
where the last line follows from the matrix inversion lemma, ie,
\begin{equation*}
  (L^{-1} + H^T J^{-1} H)^{-1} = L - LH^T (HLH^T + J)^{-1} HL.
\end{equation*}
Equation~\eqref{eq:new_Sigma} follows from $\Sigma_{t+1}^{-1} = q$.
The linear term is
\begin{align*}
  l &= O^T\Sigma_O^{-1}y(t+1) - F^T \Sigma_t^{-1}(G-\mu_t) +
  (1-AF)^T\Sigma_D^{-1}AG \\
  &= O^T\Sigma_O^{-1}y(t+1) + \Sigma_D^{-1}A \left( \Sigma_t^{-1} +
    A^T \Sigma_D^{-1} A \right)^{-1} \Sigma_t^{-1} \mu_t.
\end{align*}

While I could (and have with much effort) derive
Eqn.~\eqref{eq:new_mu} from the formula $\mu_{t+1} = \Sigma_{t+1} l$,
here I only verify that $q \mu_{t+1} = l$.  I will use the following
lemma:
\begin{align*}
  (q-O^T\Sigma^{-1}O)A &= \left( \Sigma_D^{-1} - \Sigma_D^{-1} A (
    \Sigma_t^{-1} + A^T \Sigma_D^{-1} A)^{-1} A^T \Sigma_D^{-1}
  \right) A \\
  &= (\Sigma_D + A \Sigma_t A^T)^{-1} A \quad \text{ by matrix inversion
    lemma} \\
  &= (A^{-1} \Sigma_D + \Sigma_t A^T)^{-1} \\
  &= (\Sigma_t^{-1} A^{-1} \Sigma_D + A^T)^{-1} \Sigma_t^{-1}\\
  &= \Sigma_D^{-1} (\Sigma_t^{-1} A^{-1}  + A^T\Sigma_D^{-1})^{-1} \Sigma_t^{-1}\\
  &= \Sigma_D^{-1} A (\Sigma_t^{-1}  + A^T\Sigma_D^{-1} A)^{-1} \Sigma_t^{-1}.
\end{align*}
Now the verification is simply:
\begin{align*}
  q \mu_{t+1} &= qA\mu_t + O^T \Sigma_O^{-1} \ti{y}{t+1} - O^T
  \Sigma_O O A \mu_t \\
  &= O^T \Sigma_O^{-1} \ti{y}{t+1} + (q - O^T \Sigma_O O) A \mu_t \\
  &= O^T\Sigma_O^{-1}y(t+1) + \Sigma_D^{-1}A \left( \Sigma_t^{-1} +
    A^T \Sigma_D^{-1} A \right)^{-1} \Sigma_t^{-1} \mu_t \quad \text{
    by the lemma}\\
  &= l.
\end{align*}

Using the following abbreviation and calculation (Note: $H\mu_t = G$)
\begin{align*}
  H &\equiv (\Sigma_t^{-1} + A^T \Sigma_D^{-1}A)^{-1} \Sigma_t^{-1} \\
  &= 1 - \Sigma_t A^T(A\Sigma_t A^T + \Sigma_D)^{-1} A \quad \text{ By
    matrix inversion lemma}
\end{align*}
I write the constant term as
\begin{align*}
  c &= \mu_t^T \left( (H-1)^T\Sigma_t^{-1} (H-1) + H^TA^T\Sigma_D^{-1}
    AH \right) \mu_t \\
  & \quad + \ti{y^T}{t+1} \Sigma_O^{-1}
  \ti{y}{t+1} - 2R_t + \logdet\\
  &= \mu_t^T  A^T(A\Sigma_t A^T + \Sigma_D)^{-1}A \mu_t
  + \ti{y^T}{t+1} \Sigma_O^{-1} \ti{y}{t+1} - 2R_t \\
  & \quad + \logdet,
\end{align*}
\marginpar{$-\mu_t^T\Sigma_t^{-1}\mu_t$?}and find
\begin{align*}
  R_{t+1} = R_t -\frac{1}{2} \Big( &
  \mu_t^T  A^T(A\Sigma_t A^T + \Sigma_D)^{-1}A \mu_t 
   - \mu_{t+1}^T \Sigma_{t+1}^{-1} \mu_{t+1} \\
  & + \ti{y^T}{t+1} \Sigma_O^{-1} \ti{y}{t+1} + \logdet \Big).
\end{align*}
To verify Eqn.~\eqref{eq:new_R} I'll use the following notation and
equalities:
\begin{align}
  X &\equiv \Sigma_D + A \Sigma_t A^T \\
  l &= O^T\Sigma_O^{-1} \ti{y}{t+1} + X^{-1} A \mu_t \\
  \mu_{t+1} &= \Sigma_{t+1} l \\
  \mu_{t+1}^T \Sigma_{t+1}^{-1} \mu_{t+1} &= l^T \Sigma_{t+1} l \\
  \Sigma_{t+1} &= (O^T\Sigma_O^{-1}O + X^{-1})^{-1} \\
  &= X - XO^T(OXO^T + \Sigma_O)^{-1} OX \\
  \Sigma_y &= O(A\Sigma_t A^T + \Sigma_D) O^T + \Sigma_O \\
  &= OXO^T + \Sigma_O \\
  \Sigma_y^{-1} &= \Sigma_O^{-1} - \Sigma_O^{-1} O (O^T\Sigma_O^{-1}O
  + X^{-1})^{-1}O^T \Sigma_O^{-1} \\
  &= \Sigma_O^{-1} - \Sigma_O^{-1} O \Sigma_{t+1}O^T \Sigma_O^{-1} \\
  \Delta_R &\equiv \mu_t^TA^T X^{-1} A \mu_t - \mu_{t+1}^T
  \Sigma_{t+1}^{-1} \mu_{t+1} + \ti{y}{t+1}^T \Sigma_O^{-1} \ti{y}{t+1}.
\end{align}
Using the derivation
\begin{align*}
  \Sigma_O^{-1} O \Sigma_{t+1} X^{-1} &=
  \Sigma_O^{-1}O(1-XO^T\Sigma_yO) \\
  &= \Sigma_O^{-1}(1-OXO^T\Sigma_y^{-1})O \\
  &= \Sigma_O^{-1}(\Sigma_y-OXO^T)\Sigma_y^{-1}O \\
  &= \Sigma_O^{-1}(\Sigma_O)\Sigma_y^{-1}O \\
  &= \Sigma_y^{-1}O \\
\end{align*}
and considering the three terms of
\begin{align*}
  l^T \Sigma_{t+1} l = & (O^T\Sigma_O^{-1} \ti{y}{t+1})^T \Sigma_{t+1}
  (O^T\Sigma_O^{-1} \ti{y}{t+1}) \\
  & \quad + 2(O^T\Sigma_O^{-1} \ti{y}{t+1})^T \Sigma_{t+1} X^{-1}A\mu_t \\
  & \quad + (X^{-1}A\mu_t)^T\Sigma_{t+1} X^{-1}A\mu_t \\
  \equiv & T_1+T_2+T_3
\end{align*}
separately, I find
\begin{align*}
  T_1 &= \ti{y}{t+1}^T \Sigma_O^{-1} \ti{y}{t+1} - \ti{y}{t+1}^T
  \Sigma_y^{-1} \ti{y}{t+1} \\
  T_2 &= 2 \ti{y}{t+1}^T \Sigma_y^{-1} OA\mu_t \\
  T_3 &= (A\mu_t)^T X^{-1} (A\mu_t) - (OA\mu_t)^T
  \Sigma_y^{-1}(OA\mu_t)
\end{align*}
Equation~\eqref{eq:new_R} follows pretty easily.

\section{Assignment Algorithms}
\label{sec:assignment}

\subsection{Hungarian}
\label{sec:hungarian}

I spent two weeks reading material on combinatorial optimization;
ultimately using the material on pages 201-206 of Eugene Lawler's
\emph{Combinatorial Optimization: Networks and Matroids} (Dover 2001
reprint of the 1976 Holt Rinehardt and Winston edition) to implement
the algorithm in python.  I found it helpful to think about the
problem in terms of a circuit.  For example, consider the following
problem and circuit:

Finding the optimal assignment $X$ for the utility matrix
\begin{equation*}
  w =
  \begin{bmatrix}
    4 & 5 & 8 \\
    2 & 3 & 7
  \end{bmatrix}
\end{equation*}
is equivalent to the linear programming problem (quoting Lawler)
\begin{description}
\item[Maximize:] $\sum_{i,j} w_{i,j} x_{i,j}$
\item[Subject to:]
  \begin{align*}
    \sum_i x_{i,j} &\leq 1 \\
    \sum_j x_{i,j} &\leq 1 \\
    x_{i,j} \geq 0
  \end{align*}
\end{description}
with the understanding that $x_{i,j} = 1 \implies (i,j)\in X$ and
$x_{i,j} = 0 \implies (i,j)\notin X$.
The dual linear programming problem is
\begin{description}
\item[Minimize] $\sum_i u_i + \sum_j v_j$
\item[Subject to:]
  \begin{align*}
    u_i + v_i &\geq w_{i,j} \\
    u_i & \geq 0 \\
    v_j & \geq 0
  \end{align*}
\end{description}
The following orthogonality conditions are necessary and sufficient
for optimality:
\begin{align*}
  x_{i,j} > 0 & \implies u_i + v_j = w_{i,j}\\
  u_i > 0 & \implies \sum_j x_{i,j} = 1\\
  v_j > 0 & \implies \sum_i x_{i,j} = 1
\end{align*}
In turn, the linear programming problem is equivalent to solving the
circuit in Figure~\ref{fig:assignment1}.
\begin{figure*}
  \centering
    \resizebox{.8\textwidth}{!}{\includegraphics{ha.pdf}}
  \caption{Circuit equivalent of assignment problem}
  \label{fig:assignment1}
\end{figure*}

The algorithm proceeds as follows:
\begin{enumerate}
\item Start with $X=()$ and $u= [ 8,  8] v= [ 0,  0,  0] $
\item Identify an \emph{augmenting} path, ie, a directed path that
  starts with a link with no back voltage on the diode in series with
  the voltage $w_{i,j}$ that is not in $X$ and perhaps continues with
  pairs of links the first of which is in $X$ and goes from some $v_j$
  to a $u_i$ followed by a link not in $X$ with zero voltage on the
  diode that goes from a $u_i$ to a $v_j$.  The diodes and voltages
  allow adding an augmenting forward current of value 1 through such a
  path.  The augmenting current removes from $X$ those links in the
  path that were in $X$ and appends to $X$ those in the path that were
  not in $X$ before.  For the example, the first augmenting path is
  $[(1,0)]$.  The augmentation yields $X=[(1,0)]$.
\item In this step, I want to adjust some $u$ and $v$ values;
  decreasing the $u$s and increasing the $v$s.  I select the
  adjustment value $\delta$ by assigning candidates $\pi$ to each
  node $j\in T$.  The value of $\pi_j$ is the minimum amount by which a
  link to $j$ that is not in $X$ is out of kilter.  For this step, the
  $\pi$ values are $[ 1, 5, 6]$, $\delta=1$, and the adjustment yields
  $u= [ 7, 8],~ v= [ 0, 0, 0],~ \pi= [ 0, 4, 5]$
\item With the adjusted $u_0$ value I recalculate and find
  $\pi=[0,3,4]$.  Now all links to the node $j=0$ are in kilter.  I
  can increase $v_0$ by any $\delta$ without throwing those links out
  of kilter if I also decrease the values of $u_i$ at the other ends
  of those links by the same amount.  An adjustment by $\delta=3$
  yields $u= [ 4, 5],~ v= [ 3, 0, 0],~ \pi= [ 0, 0, 1]$
\item These voltages enable the augmenting path $[(1,1) (1,0) (0,0)]$
  which yields $X= [(0, 0) (1, 1)]$
\item End.  With these voltages and currents there would be no current
  through the $u$ voltage sources; I imagine removing them.
\end{enumerate}

\begin{verbatim}
Begin step 1.3: Find pi by searching unscanned_s and links not in X.
  unscanned_S= [0, 1]
  u= [ 8.  8.]
  v= [ 0.  0.  0.]
 pi= [ 80.  80.  80.]
Found pi= [ 4.  3.  0.]

Since pi[2]==0 and 2 is not in X_T, start step 2 augmentation from j=2
  with X= 
 S_label= {0: None, 1: None} T_label= {0: 0, 1: 0, 2: 0}
 Augmenting path= (0,2) 
 After step 2 augmentation:  X= (0, 2) Y= 
              and unscanned_S= [1]

Begin step 1.3: Find pi by searching unscanned_s and links not in X.
  unscanned_S= [1]
  u= [ 8.  8.]
  v= [ 0.  0.  0.]
 pi= [ 80.  80.  80.]
Found pi= [ 6.  5.  1.]

Before step 3, delta=1.000:
  u= [ 8.  8.] v= [ 0.  0.  0.] pi= [ 6.  5.  1.]
  subtract from u[i] for i in [1]
  subtract from pi[j]>0 otherwise add to v[j]
 After step 3 adjustment:
  u= [ 8.  7.] v= [ 0.  0.  0.] pi= [ 5.  4.  0.]
  New scannable T nodes: 2 
  S_label= {1: None} T_label= {0: 1, 1: 1, 2: 1}

Begin step 1.3: Find pi by searching unscanned_s and links not in X.
  unscanned_S= []
  u= [ 8.  7.]
  v= [ 0.  0.  0.]
 pi= [ 5.  4.  0.]
Found pi= [ 5.  4.  0.]

Since pi[2]==0 and X_T[2]=0, set S_label[0] = 2, mark S[0] unscanned
and mark T[2] scanned

Begin step 1.3: Find pi by searching unscanned_s and links not in X.
  unscanned_S= [0]
  u= [ 8.  7.]
  v= [ 0.  0.  0.]
 pi= [ 5.  4.  0.]
Found pi= [ 4.  3.  0.]

Before step 3, delta=3.000:
  u= [ 8.  7.] v= [ 0.  0.  0.] pi= [ 4.  3.  0.]
  subtract from u[i] for i in [0, 1]
  subtract from pi[j]>0 otherwise add to v[j]
 After step 3 adjustment:
  u= [ 5.  4.] v= [ 0.  0.  3.] pi= [ 1.  0.  0.]
  New scannable T nodes: 1 
  S_label= {0: 2, 1: None} T_label= {0: 0, 1: 0, 2: 1}

Begin step 1.3: Find pi by searching unscanned_s and links not in X.
  unscanned_S= []
  u= [ 5.  4.]
  v= [ 0.  0.  3.]
 pi= [ 1.  0.  0.]
Found pi= [ 1.  0.  0.]

Since pi[1]==0 and 1 is not in X_T, start step 2 augmentation from j=1
  with X= (0, 2) 
 S_label= {0: 2, 1: None} T_label= {0: 0, 1: 0, 2: 1}
 Augmenting path= (0,1) (0,2) (1,2) 
 After step 2 augmentation:  X= (0, 1) (1, 2) Y= 
              and unscanned_S= []

Finished! delta=80.000 > min_u=4.000
u= [ 5.  4.] v= [ 0.  0.  3.] pi= [ 80.  80.  80.]
\end{verbatim}

\subsection{Variation on Hungarian Algorithm}
\label{sec:variation}

For my application, I want allow any number of hits to be explained by
false alarms or new targets, which corresponds to relaxing the
constraint $\sum_i x_{i,j} \leq 1$ for two values of $j$.  I need a
variant of the Hungarian algorithm that recognizes ${\cal J} \subset
T$ a subset of nodes with \emph{unlimited capacity}.  To illustrate, I
modify the problem in the previous subsection by making ${\cal J} =
\left\{2\right\}$:
\begin{description}
\item[Maximize:] $\sum_{i,j} w_{i,j} x_{i,j}$
\item[Subject to:]
  \begin{align*}
    \sum_i x_{i,j} &\leq 1 ~\forall i\\
    \sum_j x_{i,j} &\leq 1  ~\underline{\forall j \notin {\cal J}} \\
    x_{i,j} \geq 0
  \end{align*}
With the same utility matrix:
\begin{equation*}
  w =
  \begin{bmatrix}
    2 & 3 & 7 \\
    4 & 5 & 8
  \end{bmatrix}
\end{equation*}
\end{description}
The only required modifications of the Hungarian algorithm are for
$j\in {\cal J}$ in the following steps:
\begin{itemize}
\item In augmentation processing, put the link $(i,j)$ in a special
  set $Y$ and henceforth consider the node $i$ as \emph{covered}, ie,
  as if it were in $X$.  It is an error if any link to $j$ ever
  appears in $X$.
\item In adjusting the dual variables for step 3, set $v_j$ and
  $\pi_j$ to zero.
\item When finished report both $X$ and $Y$
\end{itemize}

The circuit corresponding to this problem appears in
Figure~\ref{fig:assignment2}
\begin{figure*}
  \centering
    \resizebox{.8\textwidth}{!}{\includegraphics{ha2.pdf}}
  \caption{Circuit equivalent of the assignment problem with no
    restriction on the number of $i$ nodes assigned to node $j=0$.}
  \label{fig:assignment2}
\end{figure*}

\subsection{Murty's Algorithm}
\label{sec:murty}

I like the explanation of the algorithm in Murty's
paper\footnote{\emph{An Algorithm for Ranking all the Assignments in
    Order of Increasing Cost}, Katta G.  Murty Operations Research,
  Vol. 16, No. 3 (May - Jun., 1968), pp.  682-687.}  Murty defines a
collection of assignments or solutions that \emph{includes} a set of
links $I$ and excludes another set of links $E$ as the
\underline{node} $N(I,E)$.  To avoid confusion with the vertices or
nodes of the Hungarian algorithm, will underline Murty's concept of
\underline{node}.  I have modified the algorithm for vertices with
unlimited capacity as follows: After appending a link $(i,j)$ to the
list $I$ for \underline{node} $N(I,E)$, Murty's algorithm strikes the
$i^{\text th}$ row and $j^{\text th}$ column of the utility matrix
$w$.  However, if $j\in {\cal J}$, I do not strike the $j^{\text th}$
column.

\section{Filtering}
\label{sec:filtering}

In this appendix I have two subsections that I wrote before I
realized that filtering is inappropriate for tracking.  In
\ref{sec:filtering1} I give a general recursive algorithm for filtering
for any state space model.  In \ref{sec:filter2} I discuss the
difficulty of filtering for the model of Section~\ref{sec:model1}

\subsection{General Filtering}
\label{sec:filtering1}

With the definitions
\begin{align}
  \label{def:alpha}
  \alpha_t &\equiv P(\ti{s}{t}|\ts{y}{\tau}{1}{t}) \text{ The updated distribution} \\
  f_t &\equiv P(\ti{s}{t}|\ts{y}{\tau}{1}{t-1}) \text{ The forecast
    distribution} \\
  \gamma_t &\equiv P(\ti{y}{t}|\ts{y}{\tau}{1}{t-1}) \text{ The
    incremental likelihood}
\end{align}
and applying a model characterized by the distributions
\begin{align}
  \label{def:alpha0}
  &\alpha_0  &&\text{The state prior} \\
  &P(\ti{s}{t+t}|\ti{s}{t}) &&\text{The state transition probability} \\
  &P(\ti{y}{t}|\ti{s}{t}) &&\text{The conditional observation probability}
\end{align}
I can write recursive filtering as follows:
\begin{align}
  f_t &= \EV{\alpha_{t-1}} {P(\ti{s}{t}|\ti{s}{t-1})} \\
  \gamma_t &= \EV{\ti{f}{t}} {P(\ti{y}{t}|\ti{s}{t})} \\
  \alpha_t &= \frac{f_t P(\ti{y}{t}|\ti{s}{t})}{\gamma_t} \\
\end{align}

\subsection{Filtering for Model One}
\label{sec:filter2}

With luck (I have not been so lucky with this application) one might
find a parametric form for the state prior $\alpha(0)$ (see
Eqn.~\eqref{def:alpha0}) that, combined with $\ti{y}{1}$, yields an
expression for $\alpha(1)$ that has the same parametric form.  Such a
form is called a \emph{conjugate family}.  Two particularly simple
cases are discrete hidden Markov models and Kalman filters.

Here, I will analyze the use of a Gaussian for $\alpha(0)$.  From that
choice it follows that each subsequent $\ti{\alpha}{t}$ is much more
complex.  Thus any actual implementation that starts with such an
$\alpha(0)$ must use simplifying approximations.  I will conclude the
section by considering a few such approximations.

Let me suppose that the initial state distribution is composed of
independent Gaussians, ie,
\begin{align*}
  \ti{\alpha}{0} &= P_{\os{S}{0,j}{j=1}{N} }\\
  &= \prod_{j=1}^{N} P_{\ti{S}{0,j}} \\
  &= \prod_{j=1}^{N} \normal{\mu_j}{\Sigma_j}.
\end{align*}
To find the forecast $\ti{f}{1}$, I can apply the state dynamics
\eqref{eq:dynamics} to each object independently with the result
\begin{align}
  \tilde \mu_j &= A \mu_j \\
  \tilde \Sigma_j &= A \Sigma_j A\transpose + \Sigma_D \\
  \label{eq:defF}
  (\tilde \mu_j, \tilde \Sigma_j ) &\equiv F(\mu_j,\Sigma_j) \\
  \label{eq:f1}
  \ti{f}{1} &= \prod_{j=1}^{N} {\cal N}(F(\mu_j,\Sigma_j)),
\end{align}
where Eqn.~\eqref{eq:defF} defines the map $F$ from a distribution at
time $t$ to a distribution at time $t+1$.  Note that like
$\ti{\alpha}{0}$, $\ti{f}{1}$ is simply Gaussian.  On the other hand
\begin{equation}
  \label{eq:a1}
  a(1) \equiv f(1) P_{y|S} = \frac{1}{\left| \M \right|} \sum_{M \in \M}
  \prod_{i=1}^N P(y(t,M(i)|s(t,i)) \cdot \left. {\cal
      N}(F(\mu_j,\Sigma_j))\right|_{s_i}
\end{equation}
which provides an unnormalized version of the updated distribution
$\alpha(1)$ has at least two unfortunate properties:
\begin{enumerate}
\item The distribution of states is a sum of $\left| \M \right|$
  Gaussians.  If you want to track $N=100$ objects, $\alpha(1)$ will
  have $\left| \M \right| = N! \approx 10^{156}$ terms.
\item For each term in the sum, the distributions of the separate
  objects $s_j$ are independent, but that independence property does
  not hold for the sum.  So it is not true that
  \begin{equation*}
    P(s(1)|y(1)) = \prod_{j=1}^N  P\left( \left( s(1,j) \right)|y(1)
    \right).
  \end{equation*}
\end{enumerate}

Leveraging the notational clarity of \eqref{eq:a1}, I can also write
\begin{align*}
  \ti{\alpha}{T} \propto \sum_{\os{M}{t}{t=1}{T}:\ti{M}{t}\in \M}
  \prod_{i=1}^N& \left[
  P\left( \os{y}{\ti{M}{t,i}}{t=1}{T} | \os{s}{t,i}{t=1}{T}\right) \right. \\
  & \left. P\left( \os{s}{t,i}{t=1}{T}| \ti{s}{0,i} \right)
  P\left( \ti{s}{0,i} \right) \right] .
\end{align*}
Since
\begin{equation*}
  P\left( \os{y}{\ti{M}{t,i}}{t=1}{T} | \os{s}{t,i}{t=1}{T}\right) =
  \prod_{t=1}^T P\left( \ti{y}{t,\ti{M}{t,i}} | \ti{s}{t,i}\right)
\end{equation*}
and
\begin{equation*}
  P\left( \os{s}{t,i}{t=1}{T}| \ti{s}{0,i} \right) = \prod_{t=1}^T
  P\left( \ti{s}{t,i} | \ti{s}{t-1,i}\right),
\end{equation*}
\begin{equation}
  \label{eq:alphaT}
  \ti{\alpha}{T} \propto \sum_{\os{M}{t}{t=1}{T}}
  \prod_{i=1}^N  P\left( \ti{s}{0,i} \right) \prod_{t=1}^T
  P\left( \ti{y}{t,\ti{M}{t,i}} | \ti{s}{t,i}\right)
  P\left( \ti{s}{t,i} | \ti{s}{t-1,i}\right).
\end{equation}
Like \eqref{eq:a1} \eqref{eq:alphaT} is a mixture of Gaussians.  The
number of components in the mixture $\left| \ts{M}{t}{1}{T}\right| =
(N!)^T$ is absurdly large.

\vfill \hrule To checkout: \emph{ svn --username you --password yours
  co http://fraserphysics.com/ps}
\begin{verbatim}
$Id$
\end{verbatim}

\end{document}

%%%---------------
%%% Local Variables:
%%% eval: (load-file "SeqKeys.el")
%%% eval: (TeX-PDF-mode)
%%% End:

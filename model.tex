\documentclass[12pt]{article}

\usepackage{amsmath,amsfonts}
\usepackage{graphicx}
\usepackage{showlabels}

\newcommand{\normal}[2]{{\cal N}(#1,#2)}
\newcommand{\NormalE}[3]{{\mathcal{N}}\left.\left(#1,#2\right)\right|_{#3}}
\newcommand{\xdot}{{\dot x}}
\renewcommand{\th}{^{\text{th}}}
\newcommand{\field}[1]{\mathbb{#1}}
\newcommand{\REAL}{\field{R}}
\newcommand{\RATIONAL}{\field{Q}}
\newcommand{\INTEGER}{\field{Z}}
\newcommand{\EV}[2]{\field{E}_{#1}\left[#2\right]}
\newcommand{\M}{{\cal M}}
\newcommand{\transpose}{^\top}
\newcommand{\os}[4]{{\left[ #1(#2) \right]}_{#3}^{#4}} % Object sequence
\newcommand{\ti}[2]{{#1}{(#2)}}                         % Index
\newcommand{\ts}[4]{\os{#1}{#2}{#2=#3}{#4}} % Time series
%\newcommand{\ts}[4]{{\left[ #1(#2) \right]}_{#2=#3}^{#4}} % Sequence
\newcommand{\argmin}{\operatorname*{argmin}}
\newcommand{\argmax}{\operatorname*{argmax}}
\newcommand{\cS}{{\cal S}}
\newcommand{\logdet}{\log\left(\left|\Sigma_D\right| \left| \Sigma_O
    \right| \right)}

\title{Model Based Motion Detection}
\author{Andy Fraser}

\begin{document}
\maketitle

\section{Introduction}
\label{sec:introduction}

For tracking, I want the maximum a posteriori (MAP) or \emph{decoded}
track ie,
\begin{align*}
  \ts{\hat s}{\tau}{1}{T} &\equiv \argmax_{\ts{s}{\tau}{1}{T}}
  P(\ts{s}{\tau}{1}{T}|\ts{y}{\tau}{1}{T}) \\
  &= \argmax_{\ts{s}{\tau}{1}{T}} P(\ts{s}{\tau}{1}{T},\ts{y}{\tau}{1}{T}).
\end{align*}

Note that the MAP track is different than the sequence of
\emph{filtered} forecasts.  A filtered forecast, ie,
\begin{equation*}
  P(S(t)|\ts{y}{\tau}{1}{t-1}).
\end{equation*}
is the entire a posteriori distribution at each time given the
observations up to that time.  When I began thinking about tracking
and writing this document, I thought that the best approach would be
to do filtering.  I've left some of the work I did while thinking
about filtering in the document, but the essential material concerns
decoding.  The material on decoding does not depend on the material on
filtering.  So, you may ignore the filtering sections.

\subsection{Decoding}
\label{sec:decoding}

I will use the following definitions to describe decoding:
\begin{align*}
  u(\ts{s}{\tau}{1}{t}) & \quad \text{Utility of state sequence }
  \ts{s}{\tau}{1}{t}\\
  & \quad \equiv \log \left( P(\ts{y}{\tau}{1}{t},\ts{s}{\tau}{1}{t} \right)
  \\
  \nu(s,t) & \quad \text{Utility of best sequence ending with }
  \ti{s}{t} = s \\ 
  &  \quad \equiv \max_{\ts{s}{\tau}{1}{t}:\ti{s}{t}=s} u(\ts{s}{\tau}{1}{t}) \\
  u'(s,s',t) & \quad \text{Utility of best sequence ending with }
  \ti{s}{t},\ti{s}{t+1} = s,s' \\
  &  \quad \equiv \max_{\ts{s}{\tau}{1}{t+1}:\ti{s}{t}=s \&\ti{s}{t+1}=s'}
  u(\ts{s}{\tau}{1}{t+1}) \\
  B(s,t) & \quad \text{Best predecessor state given } \ti{s}{t+1}=s   
\end{align*}
I can evaluate the functions $B(*,t)$ and $\nu(*,t)$ recursively as
follows:
\begin{align*}
  u'(s,s',t) &= \nu(s,t) + \log\left( P_{\ti{s}{t+1}|}(s'|s) \right) +
  \log\left( P_{\ti{y}{t+1}|\ti{s}{t+1}}(\ti{y}{t+1}|s') \right) \\
  B(s,t) &= \argmax_{s'} u'(s',s,t)) \\
  \nu(s,t+1) &= u'(B(s,t),s,t)
\end{align*}
Given the functions $B(*,t)$ and $\nu(*,t)$ $\forall
t\in[1,\ldots,T]$, the following procedure decodes the best sequence
of states $ \ts{\hat s}{\tau}{1}{T}$ from a sequence of observations $
\ts{y}{\tau}{1}{T}$
\begin{align*}
  {\ti{{\hat s}}{T}} &= \argmax_s \nu(s,T) \\
  & \text{for } t \text{ from } T-1 \text{ to } 1: \\
  & \quad \ti{\hat s}{t} = B( \ti{\hat s}{t+1},t)
\end{align*}

\subsection{Filtering}
\label{sec:filtering}

With the definitions
\begin{align}
  \label{def:alpha}
  \alpha_t &\equiv P(\ti{s}{t}|\ts{y}{\tau}{1}{t}) \text{ The updated distribution} \\
  f_t &\equiv P(\ti{s}{t}|\ts{y}{\tau}{1}{t-1}) \text{ The forecast
    distribution} \\
  \gamma_t &\equiv P(\ti{y}{t}|\ts{y}{\tau}{1}{t-1}) \text{ The
    incremental likelihood}
\end{align}
and applying a model characterized by the distributions
\begin{align}
  \label{def:alpha0}
  &\alpha_0  &&\text{The state prior} \\
  &P(\ti{s}{t+t}|\ti{s}{t}) &&\text{The state transition probability} \\
  &P(\ti{y}{t}|\ti{s}{t}) &&\text{The conditional observation probability}
\end{align}
I can write recursive filtering as follows:
\begin{align}
  f_t &= \EV{\alpha_{t-1}} {P(\ti{s}{t}|\ti{s}{t-1})} \\
  \gamma_t &= \EV{\ti{f}{t}} {P(\ti{y}{t}|\ti{s}{t})} \\
  \alpha_t &= \frac{f_t P(\ti{y}{t}|\ti{s}{t})}{\gamma_t} \\
\end{align}

\section{Model Version One}
\label{sec:model1}

In this section I describe the basic model \emph{MVI}.  I'll develop
algorithms and write code for \emph{MVI} for the baseline prototype of
the watcher project.  Later, I will compare proposed enhancements
against the performance of \emph{MVI}.

The number of moving objects is a constant $N$ over time.  At each
time $t$, the separate objects $\ti{s}{t,j}$ that constitute the state
evolve independently of each other.  The following equations describe
$P(\ti{s}{t+1,j}|\ti{s}{t,j})$, ie, the dynamics of each object:
\begin{subequations}
  \label{eq:dynamics}
  \begin{align}
    \ti{s}{t+1,j} &= A  \ti{s}{t,j} + \ti{\epsilon}{t,j}, &
    \ti{\epsilon}{t,j} &\sim \normal{0}{\Sigma_{D}} \text{ iid} \\
    A &= \begin{bmatrix}
      1 & 0 & 1 & 0 \\
      0 & 1 & 0 & 1 \\
      0 & 0 & 1 & 0 \\
      0 & 0 & 0 & 1
    \end{bmatrix} &
    \Sigma_{D} &= \begin{bmatrix}
      \sigma^2_{xD} & 0 & 0 & 0 \\
      0 & \sigma^2_{xD} & 0 & 0 \\
      0 & 0 & \sigma^2_{\xdot D} & 0 \\
      0 & 0 & 0 & \sigma^2_{\xdot D}
    \end{bmatrix}
  \end{align}
\end{subequations}
Each object has the following components at time $t$:
\begin{align*}
  &&s_{0}(t,j) & \text{ horizontal position} \\
  &&s_{1}(t,j) & \text{ vertical position} \\
  &&s_{2}(t,j) & \text{ horizontal velocity} \\
  &&s_{3}(t,j) & \text{ vertical velocity}
\end{align*}
In addition to the moving objects, a complete description of a state
$\ti{s}{t}$ includes a map $\ti{M}{t}\in \M$, where $\M$ is the set of
permutations of $N$ items.  I assume that $\ti{M}{t}$ is distributed
uniformly and independently of everything else, ie,
\begin{equation*}
  P(\ti{M}{t}) = \frac{1}{\left| \M\right|} =  \frac{1}{N!}.
\end{equation*}

An observation vector consists of locations
$\ti{y}{t}=\os{y}{t,i}{i=1}{N}$.  The probability that state
$\ti{s}{t}$ would produce observation $\ti{y}{t}$ is\footnote{This
  observation model describes indistinguishable observations.  An
  easier alternative would be objects that had different colors, or
  even easier, objects that had observable unique tags.}
\begin{equation}
  \label{eq:ob}
  P(\ti{y}{t}|\ti{s}{t}) =
    \frac{1}{\left|\M \right|} \sum_{M \in \M}
    \prod_{i=1}^{N} P(\ti{y}{t,M(i)}|\ti{s}{t,i}).
\end{equation}
with normal conditional observation distributions
\begin{equation*}
  \ti{y}{t,i}|\ti{s}{t,j} \sim
  \begin{cases}
    \NormalE{O\ti{s}{t,j}}{\Sigma_o}{\ti{y}{t,i}} & i = M(j) \\
    \text{Uniform} & \text{Otherwise}
  \end{cases}
\end{equation*}
where each mean comes from the projection
\begin{equation*}
  O = \begin{bmatrix} 1 & 0 & 0 & 0\\
    0 & 1 & 0 & 0 \end{bmatrix}
\end{equation*}
and each covariance is
\begin{equation*}
  \Sigma_o = \begin{bmatrix} \sigma_{xo}^2 & 0 \\ 0 &
    \sigma_{xo}^2 \end{bmatrix}.
\end{equation*}
In summary, aside from the initial state distribution, the model has
the following 3 degrees of freedom:
\begin{center}
  \begin{tabular}{|cp{15em}c|}
    \hline
    Symbol & Description & Degrees of freedom \\
    \hline
    $\Sigma_D$ & Dynamical noise & 2 \\
    $\Sigma_o$ & Observation noise & 1 \\
    \hline
  \end{tabular} 
\end{center}

\subsection{Decoding}
\label{sec:decode1}

A solution to the decoding problem
\begin{align*}
  \ts{\hat s}{\tau}{1}{T} &\equiv \argmax_{\ts{s}{\tau}{1}{T}}
  P(\ts{s}{\tau}{1}{T}|\ts{y}{\tau}{1}{T}) \\
            &=  \argmax_{\ts{s}{\tau}{1}{T}} P(\ts{s}{\tau}{1}{T},\ts{y}{\tau}{1}{T})
\end{align*}
contains only a single sequence of permutations $\ts{\hat
  M}{t}{1}{T}$.  Thus I need not approximate a mixture of Gaussians.

\subsubsection{Single Sequence}
\label{sec:single-sequence}

I'll begin by considering decoding a trajectory of a single object
from a single sequence of observations.  Using the notation in
Section~\ref{sec:decoding}, and assuming that $\nu(s,t)$ is quadratic,
I can write the following recursion:
\begin{align}
  \nu(s,t) &= -\frac{1}{2}(s-\mu_{t})^T
    \Sigma_{t}^{-1}(s-\mu_{t}) + R_t \nonumber \\
  \label{eq:decode_u'}
  u'(s,s',t) &= \nu(s,t) -\frac{1}{2} \logdet  - 
  \frac{1}{2}(s'-As)^T \Sigma_{D}^{-1} (s'-As)  \nonumber \\
  &\quad - \frac{1}{2}(\ti{y}{t+1} - O s')^T \Sigma_{O}^{-1}(\ti{y}{t+1}
    - O s') \\
  B(s,t) &= \argmax_{q} u'(q,s,t) \nonumber \\
  \label{eq:decode_B}
  &= \left( \Sigma_t^{-1} + A^T \Sigma_D^{-1} A \right)^{-1} \left(
    \Sigma_t^{-1} \mu_t + A^T \Sigma_D^{-1} s \right) \\
  \nu(s,t+1) &= u'(B(s,t),s,t) \nonumber \\
  &\equiv  -\frac{1}{2}(s-\mu_{t+1})^T
  \Sigma_{t+1}^{-1}(s-\mu_{t+1}) + R_{t+1} \nonumber \\
  \label{eq:new_Sigma}
  \Sigma_{t+1}^{-1} & = O^T\Sigma_O^{-1} O + \left( \Sigma_D + A \Sigma_t
    A^T \right)^{-1} \\
  \label{eq:new_mu}
  \mu_{t+1} &= A \mu_t + \Sigma_{t+1} O^T \Sigma_O^{-1} \Delta_y \\
  \label{eq:new_R}
  R_{t+1} &= R_t -\frac{1}{2} \left( \Delta_y^T \Sigma_y^{-1} \Delta_y
    + \logdet \right),
\end{align}
where
\begin{align}
  \label{def:Delta_y}
  \Delta_y &\equiv \ti{y}{t+1} - OA \mu_t \\
  \label{def:Sigma_y}
  \Sigma_y &\equiv O(A\Sigma_t A^T + \Sigma_D)O^T + \Sigma_O \\
  \label{eq:Sigma_yI}
  \Sigma_y^{-1} &= \Sigma_O^{-1} - \Sigma_O^{-1} O \Sigma_{t+1} O^T
  \Sigma_O^{-1}.
\end{align}
While Defs.~\eqref{def:Delta_y} and  \eqref{def:Sigma_y} are simply
abbreviations, Eqn.~\eqref{eq:Sigma_yI} requires effort to derive.

I derive \eqref{eq:decode_B} by
solving\footnote{The derivation of \eqref{eq:decode_B}:
  \begin{align*}
    \frac{d u'(q,s,t)}{d q} &= -\Sigma_t^{-1}(q-\mu_t) + A^T \Sigma_D^{-1}
    (s - Aq) \\
    &= \Sigma_t^{-1}\mu_t + A^T \Sigma_D^{-1} s -(\Sigma_t^{-1} +
    A^T\Sigma_D^{-1}A)q \\
    q &= \left( \Sigma_t^{-1} + A^T\Sigma_D^{-1}A\right)^{-1} \left(
      \Sigma_t^{-1}\mu_t + A^T \Sigma_D^{-1} s \right)
  \end{align*}
}
\begin{equation*}
  \frac{d u'(q,s,t)}{d q} = 0.
\end{equation*}
Note that the independent variable $s$ in \eqref{eq:decode_B} is the
state at the future time $t+1$, while in \eqref{eq:decode_u'}, $s$ is
the state at the earlier time $t$.  See Appendix~\ref{app:decode} for
the derivation of \eqref{eq:new_Sigma}, \eqref{eq:new_mu}, and
\eqref{eq:new_R}.  I am surprised that the resulting recursion for
$\nu(s,t)$ is exactly Kalman filtering.

For efficiency, use the following procedure to implement the forward
recursion:
\begin{description}
\item[Calculate the state forecast mean and covariance:]
  \begin{align*}
    \mu_a &= A\mu_t \\
    \Sigma_a &= A \Sigma_t A^T + \Sigma_D
  \end{align*}
\item[Calculate the inverse covariance of the forecast observation:]
  \begin{equation*}
   \Sigma_y^{-1} = \left( O\Sigma_a O^T + \Sigma_O \right)^{-1}
  \end{equation*}
\item[Calculate the Kalman gain matrix:]
  \begin{equation*}
    K = \Sigma_a O^T \Sigma_y^{-1}
  \end{equation*}
\item[Calculate the forecast error:]
  \begin{equation*}
    \Delta_y = \ti{y}{t+1} - O\mu_a
  \end{equation*}
\item[Calculate the updated mean and covariance:]
  \begin{align*}
    \mu_{t+1} &= \mu_a +  K\Delta_y \\
    \Sigma_{t+1} &= (1-KO)\Sigma_a
  \end{align*}
\item[Calculate the new $R$:]
  \begin{equation*}
    R_{t+1} = R_t - \frac{1}{2} \left(\Delta_y^T \Sigma_y^{-1}
      \Delta_y \right)
  \end{equation*}
\end{description}

\subsubsection{Multiple Objects}
\label{sec:multiple}

Consider the possible permutations at each time to be a discrete
component of the state.  Thus
\begin{equation*}
  \ti{s}{t} \in {\cal S} \equiv {\cal M} \times {\cal X}^{N},
\end{equation*}
where $N$ is the number of objects or targets, $\cal M$ is the set of
permutations of $N$ objects, and ${\cal X}$ is the state space of a
single object.  Denoting the components of a particular state as
follows: \newcommand{\bx}{{\mathbf{x}}}
\begin{align*}
  s &\equiv (M,\bx) \\
  M & \text{ is a permutation} \\
  \bx &\equiv (x_1,x_2,\ldots,x_N),
\end{align*}
I write
\begin{align*}
  u'(M,\bx,M',\bx',t) &= \nu(M,\bx) + \log \left(
    P(M',\bx'|M,\bx)\right) + \log \left(
    P(\ti{y}{t+1}|M',\bx')\right) \\
  P(M',\bx'|M,\bx) &= P(\bx'|M',M,\bx) P(M'|M,\bx) \\
  &= \frac{1}{N!} P(\bx'|\bx) \\
  u'(M,\bx,M',\bx',t) &= \nu(M,\bx) - \log(N!) + \log \left(
    P(\bx'|\bx)\right) + \log \left( P(\ti{y}{t+1}|M',\bx')\right) \\
  \nu(M,\bx,t) &= \sum_k \nu(M,x_k,t) \\
  &= \sum_k \nu(M,\bar x_k) + \sum_k \left (\nu(M,x_k,t) - \nu(M,\bar
    x_k) \right) \\
  &= \sum_k R(k,t|M) + \sum_k \left (\nu(x_k,t|M) -  R(k,t|M) \right),
\end{align*}
where $\bar x_k$ is the $x$ value that maximizes $\nu(M,x_k,t)$ for a
the best permutaion sequence ending in the given $M$.  Each $B(s,t)$
maps from $(M',s')$ at time $t+1$ to the best pair $(M,s)$ at time
$t$, and each $\nu(s,t)$ maps from a pair $(M,s)$ at time $t$ to the
utility of the best path ending at $(M,s,t)$.  While I use the values
of $\sum_k R(k,t|M)$ to determine the best predecessor permutation.
In the calculation of $R(k,t|M)$ I drop the term $\logdet$ from
Eqn.~\eqref{eq:new_R} and $\log(N!)$ because they are independent of
both $M$ and $s$.

I have implemented the following algorithm:



\subsection{Filtering}
\label{sec:filter1}

With luck (I have not been so lucky with this application) one might
find a parametric form for the state prior $\alpha(0)$ (see
Eqn.~\eqref{def:alpha0}) that, combined with $\ti{y}{1}$, yields an
expression for $\alpha(1)$ that has the same parametric form.  Such a
form is called a \emph{conjugate family}.  Two particularly simple
cases are discrete hidden Markov models and Kalman filters.

Here, I will analyze the use of a Gaussian for $\alpha(0)$.  From that
choice it follows that each subsequent $\ti{\alpha}{t}$ is much more
complex.  Thus any actual implementation that starts with such an
$\alpha(0)$ must use simplifying approximations.  I will conclude the
section by considering a few such approximations.

Let me suppose that the initial state distribution is composed of
independent Gaussians, ie,
\begin{align*}
  \ti{\alpha}{0} &= P_{\os{S}{0,j}{j=1}{N} }\\
  &= \prod_{j=1}^{N} P_{\ti{S}{0,j}} \\
  &= \prod_{j=1}^{N} \normal{\mu_j}{\Sigma_j}.
\end{align*}
To find the forecast $\ti{f}{1}$, I can apply the state dynamics
\eqref{eq:dynamics} to each object independently with the result
\begin{align}
  \tilde \mu_j &= A \mu_j \\
  \tilde \Sigma_j &= A \Sigma_j A\transpose + \Sigma_D \\
  \label{eq:defF}
  (\tilde \mu_j, \tilde \Sigma_j ) &\equiv F(\mu_j,\Sigma_j) \\
  \label{eq:f1}
  \ti{f}{1} &= \prod_{j=1}^{N} {\cal N}(F(\mu_j,\Sigma_j)),
\end{align}
where Eqn.~\eqref{eq:defF} defines the map $F$ from a distribution at
time $t$ to a distribution at time $t+1$.  Note that like
$\ti{\alpha}{0}$, $\ti{f}{1}$ is simply Gaussian.  On the other hand
\begin{equation}
  \label{eq:a1}
  a(1) \equiv f(1) P_{y|S} = \frac{1}{\left| \M \right|} \sum_{M \in \M}
  \prod_{i=1}^N P(y(t,M(i)|s(t,i)) \cdot \left. {\cal
      N}(F(\mu_j,\Sigma_j))\right|_{s_i}
\end{equation}
which provides an unnormalized version of the updated distribution
$\alpha(1)$ has at least two unfortunate properties:
\begin{enumerate}
\item The distribution of states is a sum of $\left| \M \right|$
  Gaussians.  If you want to track $N=100$ objects, $\alpha(1)$ will
  have $\left| \M \right| = N! \approx 10^{156}$ terms.
\item For each term in the sum, the distributions of the separate
  objects $s_j$ are independent, but that independence property does
  not hold for the sum.  So it is not true that
  \begin{equation*}
    P(s(1)|y(1)) = \prod_{j=1}^N  P\left( \left( s(1,j) \right)|y(1)
    \right).
  \end{equation*}
\end{enumerate}

Leveraging the notational clarity of \eqref{eq:a1}, I can also write
\begin{align*}
  \ti{\alpha}{T} \propto \sum_{\os{M}{t}{t=1}{T}:\ti{M}{t}\in \M}
  \prod_{i=1}^N& \left[
  P\left( \os{y}{\ti{M}{t,i}}{t=1}{T} | \os{s}{t,i}{t=1}{T}\right) \right. \\
  & \left. P\left( \os{s}{t,i}{t=1}{T}| \ti{s}{0,i} \right)
  P\left( \ti{s}{0,i} \right) \right] .
\end{align*}
Since
\begin{equation*}
  P\left( \os{y}{\ti{M}{t,i}}{t=1}{T} | \os{s}{t,i}{t=1}{T}\right) =
  \prod_{t=1}^T P\left( \ti{y}{t,\ti{M}{t,i}} | \ti{s}{t,i}\right)
\end{equation*}
and
\begin{equation*}
  P\left( \os{s}{t,i}{t=1}{T}| \ti{s}{0,i} \right) = \prod_{t=1}^T
  P\left( \ti{s}{t,i} | \ti{s}{t-1,i}\right),
\end{equation*}
\begin{equation}
  \label{eq:alphaT}
  \ti{\alpha}{T} \propto \sum_{\os{M}{t}{t=1}{T}}
  \prod_{i=1}^N  P\left( \ti{s}{0,i} \right) \prod_{t=1}^T
  P\left( \ti{y}{t,\ti{M}{t,i}} | \ti{s}{t,i}\right)
  P\left( \ti{s}{t,i} | \ti{s}{t-1,i}\right).
\end{equation}
Like \eqref{eq:a1} \eqref{eq:alphaT} is a mixture of Gaussians.  The
number of components in the mixture $\left| \ts{M}{t}{1}{T}\right| =
(N!)^T$ is absurdly large.

\appendix
\section{Derivation of Formulas for Decoding the Trajectory of a Single Object}
\label{app:decode}

I obtain \eqref{eq:new_Sigma} and \eqref{eq:new_mu} by expanding
$u'(B(s,t),s,t)$ and completing the square.  I'll abbreviate $B(s,t)$
with $B$, using the notation
\begin{align*}
  B &\equiv \left( \Sigma_t^{-1} + A^T \Sigma_D^{-1} A \right)^{-1}
  \left( \Sigma_t^{-1} \mu_t + A^T \Sigma_D^{-1} s \right) \\
  &= G + Fs \\
  G &\equiv \left( \Sigma_t^{-1} + A^T \Sigma_D^{-1} A \right)^{-1}
  \Sigma_t^{-1} \mu_t \\
  F &\equiv  \left( \Sigma_t^{-1} + A^T \Sigma_D^{-1} A \right)^{-1}
  A^T \Sigma_D^{-1}.
\end{align*}
I find
\begin{align*}
  \nu(s,t+1) &= -\frac{1}{2}(B-\mu_{t})^T \Sigma_{t}^{-1}
  (B-\mu_{t}) + R_t\\
  &\quad - \frac{1}{2} (s-AB)^T \Sigma_{D}^{-1} (s-AB)\\
  &\quad - \frac{1}{2}(\ti{y}{t+1}-O s)^T \Sigma_{O}^{-1}
  (\ti{y}{t+1}-Os) -\frac{1}{2} \logdet\\
  -2 \nu(s,t+1) &= (G-\mu_{t}+Fs)^T \Sigma_{t}^{-1} (G-\mu_{t}+Fs) \\
  &\quad + ((1-AF)s-AG)^T \Sigma_{D}^{-1} ((1-AF)s-AG)\\
  &\quad + (\ti{y}{t+1}-O s)^T \Sigma_{O}^{-1} (\ti{y}{t+1}-Os) -2R_t \\
  &\equiv s^T q s - 2s^T l + c.
\end{align*}
To express $\nu_{t+1}$ in terms of $\Sigma_{t+1}$, $\mu_{t+1}$, and
$R_{t+1}$, I will use the following formulas:
\begin{align*}
  \Sigma_{t+1}^{-1} &= q \\
  \mu_{t+1} &= \Sigma_{t+1} l \\
  R_{t+1} &= -\frac{1}{2} \left( c - \mu_{t+1}^T \Sigma_{t+1}^{-1}
    \mu_{t+1} \right) .
\end{align*}
The quadratic term is
\begin{align*}
  q &= F^T\Sigma_t^{-1}F + (1-AF)^T\Sigma_D^{1-}(1-AF) +
  O^T\Sigma_O^{-1}O \\
  &= O^T\Sigma_O^{-1}O + \Sigma_D^{-1} - \Sigma_D^{-1}A
  ( \Sigma_t^{-1} + A^T\Sigma_D^{-1}A )^{-1}A^T\Sigma_D^{-1}\\
  &= O^T\Sigma_O^{-1}O + (\Sigma_D + A \Sigma_t A^T)^{-1}
\end{align*}
where the last line follows from the matrix inversion lemma, ie,
\begin{equation*}
  (L^{-1} + H^T J^{-1} H)^{-1} = L - LH^T (HLH^T + J)^{-1} HL.
\end{equation*}
Equation~\eqref{eq:new_Sigma} follows from $\Sigma_{t+1}^{-1} = q$.
The linear term is
\begin{align*}
  l &= O^T\Sigma_O^{-1}y(t+1) - F^T \Sigma_t^{-1}(G-\mu_t) +
  (1-AF)^T\Sigma_D^{-1}AG \\
  &= O^T\Sigma_O^{-1}y(t+1) + \Sigma_D^{-1}A \left( \Sigma_t^{-1} +
    A^T \Sigma_D^{-1} A \right)^{-1} \Sigma_t^{-1} \mu_t.
\end{align*}

While I could (and have with much pain) derive Eqn.~\eqref{eq:new_mu}
from the formula $\mu_{t+1} = \Sigma_{t+1} l$, here I only verify that
$q \mu_{t+1} = l$.  I will use the following lemma:
\begin{align*}
  (q-O^T\Sigma^{-1}O)A &= \left( \Sigma_D^{-1} - \Sigma_D^{-1} A (
    \Sigma_t^{-1} + A^T \Sigma_D^{-1} A)^{-1} A^T \Sigma_D^{-1}
  \right) A \\
  &= (\Sigma_D + A \Sigma_t A^T)^{-1} A \quad \text{ by matrix inversion
    lemma} \\
  &= (A^{-1} \Sigma_D + \Sigma_t A^T)^{-1} \\
  &= (\Sigma_t^{-1} A^{-1} \Sigma_D + A^T)^{-1} \Sigma_t^{-1}\\
  &= \Sigma_D^{-1} (\Sigma_t^{-1} A^{-1}  + A^T\Sigma_D^{-1})^{-1} \Sigma_t^{-1}\\
  &= \Sigma_D^{-1} A (\Sigma_t^{-1}  + A^T\Sigma_D^{-1} A)^{-1} \Sigma_t^{-1}.
\end{align*}
Now the verification is simply:
\begin{align*}
  q \mu_{t+1} &= qA\mu_t + O^T \Sigma_O^{-1} \ti{y}{t+1} - O^T
  \Sigma_O O A \mu_t \\
  &= O^T \Sigma_O^{-1} \ti{y}{t+1} + (q - O^T \Sigma_O O) A \mu_t \\
  &= O^T\Sigma_O^{-1}y(t+1) + \Sigma_D^{-1}A \left( \Sigma_t^{-1} +
    A^T \Sigma_D^{-1} A \right)^{-1} \Sigma_t^{-1} \mu_t \quad \text{
    by the lemma}\\
  &= l.
\end{align*}

Using the the following abbreviation and calculation
\begin{align*}
  H &\equiv (\Sigma_t^{-1} + A^T \Sigma_D^{-1}A)^{-1} \Sigma_t^{-1} \\
  &= 1 - \Sigma_t A^T(A\Sigma_t A^T + \Sigma_D)^{-1} A \quad \text{ By
    matrix inversion lemma}
\end{align*}
I write the constant term as
\begin{align*}
  c &= \mu_t^T \left( (H-1)^T\Sigma_t^{-1} (H-1) + H^TA^T\Sigma_D^{-1}
    AH \right) \mu_t + \ti{y^T}{t+1} \Sigma_O^{-1}
  \ti{y}{t+1} - 2R_t + \logdet\\
  &= \mu_t^T  A^T(A\Sigma_t A^T + \Sigma_D)^{-1}A \mu_t
  + \ti{y^T}{t+1} \Sigma_O^{-1} \ti{y}{t+1} - 2R_t  + \logdet,
\end{align*}
and find
\begin{align*}
  R_{t+1} = R_t -\frac{1}{2} \Big( &
  \mu_t^T  A^T(A\Sigma_t A^T + \Sigma_D)^{-1}A \mu_t 
   - \mu_{t+1}^T \Sigma_{t+1}^{-1} \mu_{t+1} \\
  & + \ti{y^T}{t+1} \Sigma_O^{-1} \ti{y}{t+1} + \logdet \Big).
\end{align*}
To verify Eqn.~\eqref{eq:new_R} I'll use the following notation and
equalities:
\begin{align}
  X &\equiv \Sigma_D + A \Sigma_t A^T \\
  l &= O^T\Sigma_O^{-1} \ti{y}{t+1} + X^{-1} A \mu_t \\
  \mu_{t+1} &= \Sigma_{t+1} l \\
  \mu_{t+1}^T \Sigma_{t+1} \mu_{t+1} &= l^T \Sigma_{t+1} l \\
  \Sigma_{t+1} &= (O^T\Sigma_O^{-1}O + X^{-1})^{-1} \\
  &= X - XO^T(OXO^T + \Sigma_O)^{-1} OX \\
  \Sigma_y &= O(A\Sigma_t A^T + \Sigma_D) O^T + \Sigma_O \\
  &= OXO^T + \Sigma_O \\
  \Sigma_y^{-1} &= \Sigma_O^{-1} - \Sigma_O^{-1} O (O^T\Sigma_O^{-1}O
  + X^{-1})^{-1}O^T \Sigma_O^{-1} \\
  &= \Sigma_O^{-1} - \Sigma_O^{-1} O \Sigma_{t+1}O^T \Sigma_O^{-1} \\
  \Delta_R &\equiv \mu_t^TA^T X^{-1} A \mu_t - \mu_{t+1}^T
  \Sigma_{t+1} \mu_{t+1} + \ti{y}{t+1}^T \Sigma_O^{-1} \ti{y}{t+1}.
\end{align}
Using the derivation
\begin{align*}
  \Sigma_O^{-1} O \Sigma_{t+1} X^{-1} &=
  \Sigma_O^{-1}O(1-XO^T\Sigma_yO) \\
  &= \Sigma_O^{-1}(1-OXO^T\Sigma_y^{-1})O \\
  &= \Sigma_O^{-1}(\Sigma_y-OXO^T)\Sigma_y^{-1}O \\
  &= \Sigma_O^{-1}(\Sigma_O)\Sigma_y^{-1}O \\
  &= \Sigma_y^{-1}O \\
\end{align*}
and considering the three terms of
\begin{align*}
  l^T \Sigma_{t+1} l = & (O^T\Sigma_O^{-1} \ti{y}{t+1})T \Sigma_{t+1}
  (O^T\Sigma_O^{-1} \ti{y}{t+1}) \\
  + & 2(O^T\Sigma_O^{-1} \Sigma_{t+1} X^{-1}A\mu_t \\
  + & (X^{-1}A\mu_t)\Sigma_{t+1} X^{-1}A\mu_t \\
  \equiv & T_1+T_2+T_3
\end{align*}
separately, I find
\begin{align*}
  T_1 &= \ti{y}{t+1}^T \Sigma_O^{-1} \ti{y}{t+1} - \ti{y}{t+1}^T
  \Sigma_Y^{-1} \ti{y}{t+1} \\
  T_2 &= 2 \ti{y}{t+1}^T \Sigma_Y^{-1} OA\mu_t \\
  T_3 &= (A\mu_t)^T X^{-1} (A\mu_t) - (OA\mu_t)^T
  \Sigma_y^{-1}(OA\mu_t)
\end{align*}
Equation~\eqref{eq:new_R} follows pretty easily.

\vfill \hrule To checkout: \emph{ svn --username you --password yours
  co svn://fraserphysics.com/ps}
\begin{verbatim}
$Id$
\end{verbatim}

\end{document}

\newpage
\hrule
From here on I've just kept old text that I might use again as I
develop the document.\\
\hrule
\section{Model}
\label{sec:model}

The number of moving objects in my model is a constant $N_s$ over
time.  Objects may or may not be visible.  At time $t$, visibility,
position, and velocity
\begin{equation*}
  \ti{s}{j,t} \equiv \left(\ti{v}{j,t},\ti{x}{j,t},\ti{\dot x}{j,t} \right)
\end{equation*}
characterize the state of the $j\th$ object, and the vector
\begin{equation*}
  \ti{s}{t} \equiv \os{s}{j,t}{j=1}{N_s}
\end{equation*}
constitutes the entire state.  Each visibility component has three
possible values with the following interpretation
\begin{equation*}
  \ti{v}{j,t} =
  \begin{cases}
    1 & \text{Object is visible} \\
    2 & \text{Object is not visible for this frame} \\
    3 & \text{Object is not visible for at least three frames}
  \end{cases}
\end{equation*}
The components $\ti{s}{j,t}$ of the state evolve independently of each
other.  The following equations describe
$P(\ti{s}{j,t+1}|\ti{s}{j,t})$, ie, the dynamics of each component:
\begin{subequations}
  \label{eq:dynamics}
  \begin{align}
    P_{\ti{v}{j,t+1}|\ti{v}{j,t}} &= V, & V &=
    \begin{bmatrix}
      P(1 \rightarrow 1) & P(1 \rightarrow 2) & 0 \\
      P(2 \rightarrow 1) & P(2 \rightarrow 2) & P(2 \rightarrow 3) \\
      0 & P(3 \rightarrow 2) & P(3 \rightarrow 3)
    \end{bmatrix} \\
    \begin{bmatrix} \ti{x}{j,t+1} \\ \ti{\xdot}{j,t+1} \end{bmatrix}
    &= A   \begin{bmatrix} \ti{x}{j,t} \\ \ti{\xdot}{j,t} \end{bmatrix}
    + \epsilon_{j,t}, & \epsilon_{j,t} &\sim
    \normal{0}{\Sigma_{D}} \text{ iid} \\
    A &= \begin{bmatrix}
      1 & 0 & 1 & 0 \\
      0 & 1 & 0 & 1 \\
      0 & 0 & 1 & 0 \\
      0 & 0 & 0 & 1
    \end{bmatrix} &
    \Sigma_{D} &= \begin{bmatrix}
      \sigma^2_{xD} & 0 & 0 & 0 \\
      0 & \sigma^2_{xD} & 0 & 0 \\
      0 & 0 & \sigma^2_{\xdot D} & 0 \\
      0 & 0 & 0 & \sigma^2_{\xdot D}
    \end{bmatrix}
  \end{align}
\end{subequations}

An observation vector consists of $N_y$ locations
$\ti{y}{t}=\os{y}{t,i}{i=1}{N_y}$, and the probability that state $\ti{s}{t}$
would produce observation $\ti{y}{t}$ is
\begin{equation}
  \label{eq:ob}
  P(\ti{y}{t}|\ti{s}{t}) \equiv
  \begin{cases}
    0 & \text{if} N_y \neq N_{\text{visible}} \\
    \frac{1}{\left|\M \right|} \sum_{M \in \M}
    \prod_{i=1}^{N_y} P(\ti{y}{t,M(i)}|\ti{s}{t,i}) & \text{otherwise}
  \end{cases}
\end{equation}
where $\M$ is the set of permutations of $N_y$ items\footnote{This
  observation model describes indistinguishable observations.  An
  easier alternative would be objects that had different colors, or
  even easier, objects that had observable unique tags.},
\begin{equation*}
  \ti{y}{t,i}|\ti{s}{t,j} \sim
  \NormalE{\ti{x}{t,j}}{\Sigma_o}{\ti{y}{t,i}} \text{ and }
  \Sigma_o = \begin{bmatrix} \sigma_{xo}^2 & 0 \\ 0 &
    \sigma_{xo}^2 \end{bmatrix}.
\end{equation*}
In summary, aside from the initial state distribution, the model has
the following 7 degrees of freedom:
\begin{center}
  \begin{tabular}{|cp{15em}c|}
    \hline
    Symbol & Description & Degrees of freedom \\
    \hline
    $T$ & Probabilities of transition between visibility levels & 4 \\
    $\Sigma_D$ & Dynamical noise & 2 \\
    $\Sigma_o$ & Observation noise & 1 \\
    \hline
  \end{tabular} 
\end{center}

\section{Forward algorithm}
\label{sec:forward}

A complete model consists of seven parameters described in
Section~\ref{sec:model} and an initial distribution over states, ie,
$P_{s(0)}$.  Given a sequence $\os{y}{t}{t=1}{T}$ of $T$ vectors of
measurements and a complete model, the forward algorithm calculates a
sequence of \emph{forecasts} $\os{f}{t}{t=1}{T}$ and \emph{updates}
$\os{\alpha}{t}{t=1}{T}$.  Each forecast $\ti{f}{t}$ characterizes the
conditional distribution of states given all measurements up to the
previous time
\begin{equation*}
  \ti{f}{t} \rightarrow P(\ti{s}{t+1}|\os{y}{\tau}{\tau=1}{t}).
\end{equation*}
and each update $\ti{\alpha}{t}$ characterizes the conditional
distribution of states given all measurements up to the present time
\begin{equation*}
  \ti{\alpha}{t} \rightarrow P(\ti{s}{t}|\os{y}{\tau}{\tau=1}{t}).
\end{equation*}
I will denote the initial distribution over states as
$\ti{\alpha}{0}$.

The forward algorithm is recursive.  Each full iteration of the
recursion uses $\ti{y}{t+1}$ and $\ti{\alpha}{t}$ to calculate
$\ti{\alpha}{t+1}$ in the following steps:
\begin{subequations}
  \label{eq:Forward}
\begin{align}
  \label{eq:f1}
  \ti{f}{t+1} &\equiv P_{\ti{s}{t+1}|\os{y}{\tau}{\tau=1}{t}} =
  \EV{\ti{\alpha}{t}}{P(\ti{s}{t+1}|\ti{s}{t})} \\ 
  \label{eq:f2}
  \ti{a}{t+1} &\equiv
  P_{\ti{y}{t+1},\ti{s}{t+1}|\os{y}{\tau}{\tau=1}{t}} = P_{y|s}
  \ti{f}{t+1} \\
  \label{eq:f3}
  \ti{\gamma}{t+1} &\equiv P(\ti{y}{t+1}|\os{y}{\tau}{\tau=1}{t}) = \EV{\ti{f}{t+1}}{P(\ti{y}{t+1}|\ti{s}{t+1})} \\
  \label{eq:f4}
  \ti{\alpha}{t+1} &= \frac{\ti{a}{t+1}}{\ti{\gamma}{t+1}}
\end{align}
\end{subequations}
where the items on the left have the following interpretations:
\begin{description}
\item[$\ti{f}{t+1}$] A distribution of states $s$
\item[$\ti{a}{t+1}$] An unnormalized distribution of states $s$
\item[$\ti{\gamma}{t+1}$] A scalar; the conditional probability of the
  observation $\ti{y}{t+1}$ given the model and the previous observations
\item[$\ti{\alpha}{t+1}$] A distribution of states $s$
\end{description}

Each forecast and update characterizes a distribution of possible
states and thus has the same structure with the following
constituents\footnote{This is wrong.  The first update operation
  produces a mixture of Gaussians with $N_s$ components, and each
  subsequent update increases the number of components by another
  factor of $N_s$.}:
\begin{subequations}
  \label{eq:psForm}
  \begin{align}
    P_j &\equiv \begin{bmatrix} P_{v_j}(1), P_{v_j}(2), P_{v_j}(3)
    \end{bmatrix}, ~\forall j\\
    \mu_j &\equiv \begin{bmatrix} \mu_{j,1} \\ \mu_{j,2} \\ \mu_{j,3} \\
      \mu_{j,4} \end{bmatrix}, ~\forall j\\
    \Sigma_j, &~~~~~\forall j
  \end{align}
\end{subequations}

Using Eqn.~\eqref{eq:dynamics}, I implement step~\eqref{eq:f1} by:
\begin{subequations}
  \label{eq:f1I}
  \begin{align}
    P_{vfj} &= P_{v\alpha j} T\\
    \mu_{fj} &= A \mu_{\alpha j}\\
    \Sigma_{fj} &= A \Sigma_{\alpha j} A\transpose + \Sigma_D
  \end{align}
\end{subequations}

It would be nice if step~\eqref{eq:f2} lead to each component say
$a_j$ being the distribution of a weighted sum of variables with
distributions $f_k$ because that will lead to $\ti{\alpha}{t+1}$ having the
form \eqref{eq:psForm}, but $a_j$ is simply a weighted sum of the
distributions $f_k$, ie, an ugly mixture of Gaussians.

Roughly:
\begin{align*}
  a(s) &= \sum_M \prod_j P(y_{M(j)}|s_j) f(s_j) \\
  a(s_j) &= \sum_i  P(y_i|s_j) f(s_j) \sum_{M:M(j)=i}  \prod_k
  P(y_{M(k)}|s_k) f(s_k)\\
  &= \sum_i  P(y_i|s_j) f(s_j) w_{i,j} \\
  &= \sum_i  w_{i,j} \NormalE{x_j}{\Sigma_{xo}}{y_i}
  \NormalE{\mu_{fj}}{\Sigma_{fj}}{\begin{bmatrix} x_j\\ \xdot_j
    \end{bmatrix}}
\end{align*}


\subsection{Fudge}
\label{sec:fudge}

The following adjustments come to mind:
\begin{description}
\item[Hard sphere:] Require $\left|\mu_{j,1:2} - \mu_{k,1:2} \right| >
  \delta,~\forall j,k$
\item[Bounded positions:] Require $0 < \mu_{j,1} <
  \text{max}_1,~\forall j$ with a similar requirement for component 2
\item[Bounded velocities:] Require $0 < \mu_{j,3} <
  \text{max}_3,~\forall j$ with a similar requirement for component 4
\item[Bounded variance] Complicated bounds on $\Sigma_{\alpha,j}$ and
  $\Sigma_{a,j}$
\end{description}

%%%---------------
%%% Local Variables:
%%% eval: (load-file "SeqKeys.el")
%%% eval: (TeX-PDF-mode)
%%% End:

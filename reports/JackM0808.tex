\documentclass[]{article}

\usepackage{amsmath,amsfonts}
\usepackage{graphicx}
%\usepackage{showlabels}

\newcommand{\normal}[2]{{\cal N}(#1,#2)}
\newcommand{\NormalE}[3]{{\mathcal{N}}\left.\left(#1,#2\right)\right|_{#3}}
\newcommand{\xdot}{{\dot x}}
\renewcommand{\th}{^{\text{th}}}
\newcommand{\field}[1]{\mathbb{#1}}
\newcommand{\REAL}{\field{R}}
\newcommand{\RATIONAL}{\field{Q}}
\newcommand{\INTEGER}{\field{Z}}
\newcommand{\EV}[2]{\field{E}_{#1}\left[#2\right]}
\newcommand{\M}{{\cal M}}
\newcommand{\transpose}{^\top}
\newcommand{\os}[4]{{\left[ #1(#2) \right]}_{#3}^{#4}} % Object sequence
\newcommand{\ti}[2]{{#1}{(#2)}}                         % Index
%\newcommand{\ts}[4]{\os{#1}{#2}{#2=#3}{#4}} % Time series
%\newcommand{\ts}[4]{{\left[ #1(#2) \right]}_{#2=#3}^{#4}} % Sequence
\newcommand{\ts}[3]{{#1}_{#2}^{#3}} % Time series
\newcommand{\argmin}{\operatorname*{argmin}}
\newcommand{\argmax}{\operatorname*{argmax}}
\newcommand{\cS}{{\cal S}}
\newcommand{\cA}{{\cal A}}
\newcommand{\cC}{{\cal C}}
\newcommand{\logdet}{\log\left(\left|\Sigma_D\right| \left| \Sigma_O
    \right| \right)}

\title{Track Improvement}
\author{Andy Fraser}

\begin{document}
\maketitle

Existing MIT-LL software derives estimated vehicle tracks from a lower
level moving object detector resulting in useful \emph{tracklets}.
The tracklets can cue attention to times when vehicles visit locations
of interest, and they can truly track vehicles in easy cases.  However
the tracking fails if a vehicle gets close to other vehicles, stops,
maneuvers erratically, or is not detected for more than five
consecutive frames.  We can improve tracking with the following
techniques:
\begin{description}
\item[Vehicle appearance model] In addition to reporting the location
  of each moving object in each frame, each tracked vehicle will have
  a model for it's appearance.  A model gives a probabilistic
  description of the relationship of the location and orientation of
  the vehicle to its appearance in a frame.  Appearance models will
  let us make better guesses about which object in each frame maps to
  which target. The models will let us use orientation information
  from the still images to estimate direction of motion, and they will
  help us detect vehicles when they are not moving.
\item[Traffic model] We can use older observations to build
  probabilistic models of vehicle motion.  The distribution of vehicle
  motion will depend on location and time (either time of day or,
  given enough historical data, time of week).  Such models can
  improve tracking by making our guesses about where a tracked vehicle
  will go in the next frame more accurate.  They will also enable us
  to better flag anomalous behavior.
\item[Models of maneuvers] With models that combine Gaussian noise
  with a linear map for target vehicle motion and a linear map from
  the vehicle \emph{state} to its appearance in a frame, MIT-LL (and
  others) can use Kalman filters for tracking.  Kalman filters are
  marvelously powerful and efficient, and we intend to continue using
  them.  However, we augment simple Kalman filters with a technique
  called \emph{interacting multiple models}, IMMs.  An IMM models a
  collection of maneuvering modes by using a different linear map to
  describe the target motion for each mode.  The technique has the
  following two classes of benefits:
  \begin{description}
  \item[Better tracking] A multi-mode model more accurately describes
    vehicle motion and consequently the resulting track estimates are
    better.
  \item[Identification of maneuver modes] Using IMMs, our tracking
    algorithm naturally estimates a sequence of modes.  That sequence,
    perhaps with some additional processing, is the information that
    goes into the track event file.
  \end{description}
\item[Soft assignment of observations to targets] In the MIT-LL
  tracking algorithm, for each new frame, each target gets assigned
  the observation that is most probable unless there is no observation
  above a certain plausibility threshold in which case the target is
  not assigned an observation.  Thus the algorithm makes hard greedy
  assignment choices as it processes each frame.  While that approach
  yields code that runs quickly, it precludes the possibility of using
  later frames to improve the assignment choices in earlier frames.
  The state of the art technique, called \emph{multi-hypothesis
    tracking}, MHT, uses information accumulated over a sequence of
  frames to make assignment decisions.  MHT operates by maintaining as
  hypotheses a collection of plausible assignment sequences.  As it
  processes new frames, some sequences branch and others become
  implausible.
\item[Stopped mode] As a simple application of the IMM idea, we plan
  to augment the single mode greedy approach of the MIT-LL tracker
  with a simple greedy tracker that has two modes, \emph{moving} and
  \emph{stopped}.  Targets in the new stopped mode will be associated
  with stationary objects that escape the moving object detector.  The
  technique will stitch together tracklet gaps that come from stops at
  intersections.
\end{description}

None of the ideas in the preceding list is particularly novel or
profound, but by integrating them we can build a persistent
surveillance analysis tool that is reliable, flexible, and extendable.
While we could buy code that uses some of the ideas, eg, the division
of BAE that was formerly Alphatech uses MHT and IMM, in the resulting
system we could neither integrate the models at different levels, nor
optimize models and algorithms for performance.  The system would be
brittle; without reliability, flexibility, or extendibility.

\section*{Progress in 2008}
\label{sec:progres}

We have developed a theoretical framework for model based tracking,
written prototype code that implements it, and tested the code on data
from an Angle Fire collect.  Our prototype can incorporate a small
variety of model types.  By using IMMs and MHT, we obtain a marginal
performance improvement over a simple greedy approach.  We expect that
margin to grow as we incorporate features that model vehicle
appearance and position and time based traffic behavior.  In addition
to being able to incorporate new model features, our approach will let
us distribute a fixed budget of computational power flexibly; applying
more power to regions and targets that a user or higher level software
indicates are more important.

\subsection*{Tracking vs filtering}
\label{sec:track-filter}

When we started work on the WATCHER project a year ago, no one on the
team had studied tracking.  Beginning with the literature (eg,
Bar-Shalom, Li, and Kirubarajan, \emph{Estimation with Applications to
  Tracking and Navigation}) we found parallels to the problem of
estimating states from observations in a dynamical system (eg, Fraser
\emph{Hidden Markov Models and Dynamical Systems}) with which we are
familiar.

Suppose that at each integer time there is an \emph{observation}
consisting of a list of detected vehicles and characterizations of
their appearance, a \emph{state} consisting of a list of target
positions, velocities, etc., and operating modes, and a nuisance
variable called an \emph{association} consisting of a map from targets
in the state to detected vehicles in the observation.  We want to know
what a sequence of observations can tell us about the sequence of
states.  Using the symbols $\ti{y}{t}$, $\ti{x}{t}$, and $\ti{A}{t}$
for the observation, state, and association respectively in the frame
at time $t$ and $\ts{y}{1}{T}$, $\ts{x}{1}{T}$, and $\ts{A}{1}{T}$ for
the respective sequences, we want to characterize
\begin{equation*}
  P(\ts{x}{1}{T}|\ts{y}{1}{T}) = \sum_{\ts{A}{1}{T}}  P(\ts{x}{1}{T},
  \ts{A}{1}{T}|\ts{y}{1}{T}).
\end{equation*}
An essential challenge is that the number of possible association
sequences $\ts{A}{1}{T}$ is enormous.  If there are $N$ targets and
observations in every frame the number of sequences is $N!^T$.
Initially, we considered an approach analogous to Kalman filtering
applied globally.  We hoped to approximate an estimate of
$P(\ti{x}{t})|\ts{y}{1}{t})$ recursively.  That approach lead to
complicated probability distributions for which simplifying
approximations seemed difficult and it provided estimates of target
locations without providing estimates of target trajectories.  As an
alternative we chose to approximate the best guess about both the
sequence of states and the sequence of associations, ie,
\begin{equation}
  \label{eq:bear}
  \ts{{\hat x}}{1}{T},\ts{{\hat A}}{1}{T} =
  \argmax_{\ts{x}{1}{T},\ts{A}{1}{T}}   P(\ts{x}{1}{T},
  \ts{A}{1}{T}|\ts{y}{1}{T}).
\end{equation}
While solving \eqref{eq:bear} exactly requires considering an enormous
number of possible association sequences, we have developed an
algorithm that uses ideas from Kalman filtering and Viterbi decoding
that yields approximate solutions with a computational cost that is
linear in the number of frames.  The procedure produces plausible
tracks, and it is easy to incorporate into the procedure model
improvements and variations like those in the foregoing list.

% \begin{verbatim}
% $Id: $
% \end{verbatim}

\end{document}

%%%---------------
%%% Local Variables:
%%% eval: (TeX-PDF-mode)
%%% End:

\documentclass{article}

%\topmargin +1in % was +27pt
%\textheight 625pt % was 505
\begin{document}
%\pagestyle{empty}
\noindent
\hfill October 15, 2007\\
\noindent
\begin{description}
\item[To:] Watcher Trackers
\item[From:] Andy Fraser
\item[Subject:] Agenda for 2007-10-16 Meeting
\item[Time and Place] 9:00-10:00 in TA-3, Building SM-40, Room W106A
\vspace{1cm}
\end{description}

\noindent
We will meet to discuss work on tracking in general and in particular
work on tracking for the watcher project.  I'll start by reviewing the
work I've done.  Then we'll talk about how to proceed.

\begin{enumerate}
\item Current status of Andy's models and algorithms:
  \begin{description}
  \item[Models:] The models that I've developed have:
    \begin{itemize}
    \item Fixed number of targets
    \item Linear Gaussian dynamics and observations
    \item Sticky observablity (missed detections)
    \item False alarms
    \end{itemize}
  \item[Algorithms:]  The code that I've written for such models have:
    \begin{itemize}
    \item Complexity of approximately $T\cdot N!\cdot N^2$ for an
      exhaustive search for the optimal solution.  Here, $T$ is the
      number of frames, and $N$ is the number of targets.
    \item Complexity of approximately $T\cdot N_A\cdot N_\epsilon^2$ for
      a search that considers only $N_A$ associations at each time
      and has only $N_\epsilon$ trajectories in a neighborhood
    \end{itemize}
  \item[Data:] I have only worked with simulated data to date
  \item[Analysis:] Spend about 20 minutes reviewing dynamic
    programming as a generalization of the Viterbi algorithm
  \end{description}
\item Work to do:
  We will discuss general approaches and issues like the importance of
  \emph{real time} results.
  \begin{description}
  \item[Models:] Must allow the number of targets to vary over time
  \item[Algorithms:] Want complexity to go as sum of cluster
    complexities not product of cluster complexities.  Tag plausible
    errors.
  \item[Data:] To work with Albuquerque data
  \item[Analysis:] Understand performance cost of non-optimal
    algorithms
  \end{description}
\end{enumerate}

\vfill \hrule
\begin{verbatim}
$Id:$
\end{verbatim}
\end{document}

%%% Local Variables:
%%% eval: (TeX-PDF-mode)
%%% End:
\documentclass{article}

%\topmargin +1in % was +27pt
%\textheight 625pt % was 505
\begin{document}
%\pagestyle{empty}
\noindent
\hfill October 17, 2007\\
\noindent
\begin{description}
\item[To:] Watcher Trackers
\item[From:] Andy Fraser
\item[Subject:] Minutes for 2007-10-16 Meeting
\vspace{1cm}
\end{description}

\noindent
\begin{description}
\item[Attendance:] S.~Brumby, D.~Eads, A.~Fraser, R.~Porter,
  E.~Rosten and J.~Theiler.
\item[Presentations:] Andy showed 1-d simulations and reviewed the
  generalization of the Viterbi algorithm to state spaces that have
  continuous components.
\item[Discussion:] We touched on many topics including:
  \begin{itemize}
  \item The challenge to develop variants of Andy's algorithms that
    will operate on big images in acceptable time
  \item The importance of including vehicle-vehicle interactions in models
  \item The possibility of using PDEs rather than finite dimensional
    maps in models
  \item The relationship between tracking algorithms that we currently
    apply to real images and model based algorithms
  \item The importance of being \emph{real time} or \emph{causal}. Ed
    is for it, Andy doesn't want to do it, Reid thinks it's too soon
  \end{itemize}
\item[Tasks:] We touched on the following list of tasks.  I include
  names of folks that I hope will work on them
  \begin{description}
  \item[Models:] Must allow the number of targets to vary over time.
    \textbf{Andy} said that he will do it \emph{tomorrow}.
  \item[Algorithms:]
    \begin{itemize}
    \item Want complexity to go as sum of cluster complexities not
      product of cluster complexities.  I hope that I can get
      \textbf{Ed} to work on integrating things like graph
      partitioning into the algorithms.
    \item Tag plausible errors.  No one even suggested an interest in
      working on this
    \end{itemize}
  \item[Data:] Work with Albuquerque data.  Reid is working on the
    data that Christie marked.  \textbf{Andy} will write code to apply
    to that data when he has algorithms that can derive tracks from it
    in less than a day
  \item[Analysis:] Understand performance cost of non-optimal
    algorithms.  While no one volunteered to study that specific
    question, \textbf{Steven} will review Andy's algorithm derivation
    for correctness.
  \end{description}
\end{description}

\vfill \hrule
\begin{verbatim}
$Id:$
\end{verbatim}
\end{document}

%%% Local Variables:
%%% eval: (TeX-PDF-mode)
%%% End: